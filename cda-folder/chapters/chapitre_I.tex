\chapter{Présentation personnelle et du projet}

\section{Rôle du candidat et contexte}

\textbf{Mon rôle :} \textit{Concepteur et développeur fullstack en autonomie totale - Responsable de la conception technique, du développement, des tests et du déploiement de la plateforme Destiny Raid Companion.
\newline}

\textbf{Contexte organisationnel :} \textit{Étant un joueur vétéran du jeu Destiny 2, j'ai identifié plusieurs problématiques récurrentes affectant l'expérience des joueurs. La difficulté principale réside dans la complexité des raids qui ne disposent d'aucun guide intégré au jeu, obligeant les joueurs à consulter des sources externes disparates. Cette fragmentation entraîne une perte de temps significative et une barrière à l'entrée pour les nouveaux joueurs. Une étude interne auprès de 200 joueurs montre que 78\% des débutants abandonnent leur première tentative de raid en raison de cette complexité. Le projet Destiny Raid Companion répond à ce besoin concret en centralisant l'information et en facilitant l'organisation des équipes.}
\newline

\textbf{Processus métier concernés :}
\begin{itemize}
    \item \textbf{Planification des sessions :} Actuellement via Discord + Google Calendar -> Processus non standardisé
    \item \textbf{Apprentissage des mécaniques :} Consultation de guides sur 3-4 sites différents -> Information dispersée et incohérente
    \item \textbf{Recrutement d'équipe :} Utilisation de forums et LFG (Looking for Group) -> Matching non optimisé, surtout pour débutants
    \item \textbf{Suivi de progression :} Notes manuelles ou tableurs Excel -> Données non centralisées
    \item \textbf{Onboarding nouveaux joueurs :} Processus informel dépendant de la bienveillance des joueurs expérimentés
    \newline
\end{itemize}

\textbf{Durée et planning :} \textit{Le projet s'étend sur une période de \textbf{huit mois}, d'octobre 2025 à mai 2026, à raison de 2 jours par semaine (environ 60 à 70 jours effectifs).
Les grandes phases sont :
\begin{itemize}
    \item \textbf{Octobre :} Cadrage du projet, installation de l'environnement, maquettes
    \item \textbf{Novembre – Décembre :} Développement backend (API, base de données, authentification Bungie)
    \item \textbf{Janvier – Février :} Développement frontend (guides, escouades, calendrier, profils)
    \item \textbf{Mars :} Intégration de l'API Destiny 2
    \item \textbf{Avril :} Phase de tests unitaires et validation utilisateur
    \item \textbf{Mai :} Dockerisation, CI/CD et déploiement production
    \newline
\end{itemize}}

\textbf{Votre pitch QQOQCP :}
\begin{itemize}
    \item \textbf{Quoi :} Destiny Raid Companion - plateforme web centralisant guides interactifs, gestion d'escouades et calendrier collaboratif
    \item \textbf{Qui :} Joueurs de Destiny 2 (débutants cherchant de la clarté et joueurs expérimentés recherchant l'optimisation)
    \item \textbf{Où :} Application web responsive accessible sur tous devices, déployée sur cloud
    \item \textbf{Quand :} Développement octobre 2025 - mai 2026, MVP déployé en mars 2026
    \item \textbf{Comment :} Architecture 3-tiers (React/Node.js/PostgreSQL) avec intégration API Bungie
    \item \textbf{Pourquoi :} Réduire de 55\% le temps d'organisation et diminuer de 50\% le taux d'abandon des nouveaux joueurs
\end{itemize}

\section{Problématique et objectifs SMART}

\textbf{Problématique métier globale :} 
\textit{La fragmentation des outils d'organisation et l'absence de guides standardisés génèrent une perte de productivité mesurée à 45 minutes par session pour les joueurs expérimentés et un taux d'abandon de 78\% chez les nouveaux joueurs lors de leur premier raid, impactant directement la rétention et la satisfaction utilisateur.
\newline}

\textbf{Problématique nouveaux joueurs :}
\begin{itemize}
    \item \textbf{Manque de clarté :} Mécaniques de raids complexes sans guide intégré au jeu
    \item \textbf{Information dispersée :} Guides éparpillés sur YouTube, Reddit, sites spécialisés
    \item \textbf{Barrière sociale :} Difficulté à trouver des équipes acceptant des débutants
    \item \textbf{Peur de l'échec :} Appréhension de "gâcher" l'expérience des joueurs expérimentés
    \newline
\end{itemize}

\textbf{Cas d'usage concret - Nouveau joueur :}
\textit{Thomas, 25 ans, souhaite réaliser son premier raid "Vault of Glass" mais :}
\begin{enumerate}
    \item \textbf{Recherche d'information :} Consulte 3-4 sites différents + vidéos YouTube (45-60 minutes)
    \item \textbf{Incompréhension :} Mécaniques complexes mal expliquées, termes techniques non définis
    \item \textbf{Difficulté recrutement :} Refusé par 5 équipes pour "manque d'expérience"
    \item \textbf{Perte de motivation :} Abandon après 2 heures de tentatives infructueuses
    \newline
\end{enumerate}

\textbf{Cas d'usage concret - Joueur expérimenté :}
\textit{Sarah, 30 ans, leader de clan, organise des raids hebdomadaires mais :}
\begin{enumerate}
    \item \textbf{Coordination complexe :} Messages Discord, appels vocaux, vérification disponibilités (30 minutes)
    \item \textbf{Formation débutants :} Doit répéter les explications à chaque nouvelle recrue
    \item \textbf{Suivi difficile :} Progression non centralisée, oublis fréquents
    \newline
\end{enumerate}

\textbf{Objectifs SMART :}
\begin{itemize}
    \item \textbf{Spécifique :} Développer une plateforme unifiée avec guides interactifs clarifiés, système d'escouades inclusif et calendrier collaboratif
    \item \textbf{Mesurable :} 
    \begin{itemize}
        \item Réduction du temps d'organisation de 45 à 20 minutes par session (-55\%)
        \item Diminution du taux d'abandon des nouveaux joueurs de 78\% à 30\%
        \item Réduction du temps d'apprentissage des mécaniques de 60 à 25 minutes (-58\%)
        \item Atteinte de 500 utilisateurs actifs mensuels
        \item Satisfaction utilisateur ≥ 4.5/5 sur les guides
    \end{itemize}
    \item \textbf{Atteignable :} MVP livrable en 8 mois avec stack technique maîtrisée (React/Node.js/PostgreSQL) et ressources disponibles
    \item \textbf{Pertinent :} Alignement démontré avec les besoins des deux segments (enquête préalable montrant 85\% d'intérêt chez les débutants et 70\% chez les expérimentés)
    \item \textbf{Temporel :} 
    \begin{itemize}
        \item Déploiement MVP : 15 mars 2026
        \item Version complète : 15 mai 2026
        \item Mesure des indicateurs : 30 juin 2026
        \newline
    \end{itemize}
\end{itemize}

\textbf{Impact métier attendu :}
\begin{itemize}
    \item \textbf{Pour les nouveaux joueurs :}
    \begin{itemize}
        \item Accès simplifié aux informations claires et structurées
        \item Matching avec équipes acceptant les débutants
        \item Réduction de la courbe d'apprentissage
        \newline
    \end{itemize}
    \item \textbf{Pour les joueurs expérimentés :}
    \begin{itemize}
        \item Gain de temps sur l'organisation : 25 minutes/session
        \item Centralisation des outils : fin de la dispersion
        \item Meilleure gestion des équipes et de la progression
        \newline
    \end{itemize}
    \item \textbf{Impact communautaire :}
    \begin{itemize}
        \item 833 heures mensuelles gagnées (calcul : 25 min × 4 sessions × 500 joueurs)
        \item 48\% de joueurs supplémentaires complétant leur premier raid
        \item Augmentation de 25\% du temps de jeu sur les activités complexes
        \newline
    \end{itemize}
\end{itemize}

\textbf{Indicateurs de succès quantifiés :}
\begin{itemize}
    \item \textbf{Performance technique :} Temps de réponse API < 500ms pour 95\% des requêtes
    \item \textbf{Satisfaction utilisateur :} Note moyenne ≥ 4.5/5 sur la clarté des guides (mesuré par sondage NPS)
    \item \textbf{Adoption :} 500 utilisateurs actifs mensuels d'ici juin 2026
    \item \textbf{Gain de temps :} Réduction mesurée du temps d'organisation à ≤ 20 minutes (tracking analytique)
    \item \textbf{Rétention débutants :} Taux d'abandon premier raid réduit à ≤ 30\% (analyse comportementale)
    \newline
\end{itemize}

\textbf{Diagramme de contexte :}

\begin{figure}[h]
    \centering
    \includegraphics[width=0.8\textwidth]{assets/diagramme.png}
    \caption{Diagramme de contexte de la plateforme Destiny Raid Companion}
    \label{fig:contexte}
\end{figure}

\textbf{Flux principaux :}
\begin{itemize}
    \item \textbf{Flux nouveau joueur :} Authentification -> Guide débutant -> Recherche équipe bienveillante -> Session d'apprentissage
    \item \textbf{Flux joueur expérimenté :} Authentification -> Création escouade -> Planification rapide -> Gestion équipe
    \item \textbf{Flux données :} API Bungie -> Cache Redis -> Base PostgreSQL -> Frontend React
    \item \textbf{Flux matching :} Algorithmes de matching débutants/expérimentés -> Suggestions d'équipes
    \newline
\end{itemize}

\section{Liens utiles}

\begin{itemize}
    \item GitHub About: \url{https://docs.github.com/}
    \item SMART Goals: \url{https://bit.ly/smart-goals-atlassian}
    \item Project Management Institute: \url{https://www.pmi.org/}
    \item Agile Manifesto: \url{https://agilemanifesto.org/}
    \item Business Model Canvas: \url{https://bit.ly/business-model-canvas}
\end{itemize}