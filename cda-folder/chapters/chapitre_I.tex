\chapter{Présentation personnelle et du projet}

\section{Rôle du candidat et contexte}

Dans cette section, vous devez présenter votre rôle dans le projet et le contexte organisationnel. Le jury attend une explication claire de vos responsabilités et de l'environnement de travail dans lequel s'inscrit votre projet CDA.

\textbf{Votre rôle :} \textit{Concepteur et développeur fullstack en autonomie totale - Responsable de la conception technique, du développement, des tests et du déploiement de la plateforme Destiny Raid Companion.}

\textbf{Contexte organisationnel :} \textit{Le projet Destiny Raid Companion répond à un besoin métier concret : optimiser l'organisation des activités complexes de Destiny 2 qui représentent un investissement temps significatif pour la communauté. Actuellement, les processus d'organisation sont fragmentés entre plusieurs outils non intégrés, générant des inefficacités opérationnelles mesurables.}

\textbf{Processus métier concernés :}
\begin{itemize}
    \item \textbf{Planification des sessions :} Actuellement via Discord + Google Calendar → Processus non standardisé
    \item \textbf{Apprentissage des mécaniques :} Consultation de guides sur sites externes → Information dispersée
    \item \textbf{Recrutement d'équipe :} Utilisation de forums et LFG (Looking for Group) → Matching non optimisé
    \item \textbf{Suivi de progression :} Notes manuelles ou tableurs Excel → Données non centralisées
\end{itemize}

\textbf{Durée et planning :} \textit{Le projet s'étend sur une période de \textbf{huit mois}, d'octobre 2025 à mai 2026, à raison de 2 jours par semaine (environ 60 à 70 jours effectifs).
Les grandes phases sont :
\begin{itemize}
    \item \textbf{Octobre :} Cadrage du projet, installation de l'environnement, maquettes
    \item \textbf{Novembre – Décembre :} Développement backend (API, base de données, authentification Bungie)
    \item \textbf{Janvier – Février :} Développement frontend (guides, escouades, calendrier, profils)
    \item \textbf{Mars :} Intégration de l'API Destiny 2
    \item \textbf{Avril :} Phase de tests unitaires et validation utilisateur
    \item \textbf{Mai :} Dockerisation, CI/CD et déploiement production
\end{itemize}}

\textbf{Votre pitch QQOQCP :}
\begin{itemize}
    \item \textbf{Quoi :} Destiny Raid Companion - plateforme web centralisant guides interactifs, gestion d'escouades et calendrier collaboratif
    \item \textbf{Qui :} Joueurs de Destiny 2 (cibles : débutants frustrés et joueurs expérimentés recherchant l'optimisation)
    \item \textbf{Où :} Application web responsive accessible sur tous devices, déployée sur cloud
    \item \textbf{Quand :} Développement octobre 2025 - mai 2026, MVP déployé en mars 2026
    \item \textbf{Comment :} Architecture 3-tiers (React/Node.js/PostgreSQL) avec intégration API Bungie
    \item \textbf{Pourquoi :} Réduire de 50\% le temps d'organisation et diminuer de 40\% l'abandon des raids par les nouveaux joueurs
\end{itemize}

\begin{conseil}
\textbf{Ce que le jury attend dans cette section :}
\begin{itemize}
    \item Une présentation claire de votre rôle et de vos responsabilités
    \item Une explication du contexte métier et des enjeux
    \item Une justification de la pertinence du projet
    \item Un pitch QQOQCP synthétique et percutant
    \item Des indicateurs de succès mesurables
\end{itemize}

\textbf{Conseils de rédaction :}
\begin{itemize}
    \item Soyez précis sur votre fonction (évitez les généralités)
    \item Montrez votre compréhension des enjeux métier
    \item Justifiez le choix de votre projet
    \item Utilisez des chiffres et des métriques quand c'est possible
\end{itemize}
\end{conseil}

\begin{jury}
\begin{itemize}
    \item Pouvez-vous présenter votre rôle précis dans ce projet ?
    \item Quel est le contexte métier de votre entreprise ?
    \item Quels sont les enjeux techniques principaux ?
    \item Comment mesurez-vous le succès de votre projet ?
    \item Quelles sont les contraintes temporelles et budgétaires ?
\end{itemize}
\end{jury}

\section{Problématique et objectifs SMART}

Dans cette section, vous devez identifier clairement la problématique que votre projet résout et définir des objectifs SMART mesurables. Le jury attend une analyse précise des enjeux et des bénéfices attendus.

\textbf{Cas d'usage concret identifié :}
\textit{Un nouveau joueur souhaitant réaliser son premier raid "Vault of Glass" doit actuellement :}
\begin{enumerate}
    \item Consulter 3-4 sites différents pour comprendre les mécaniques (45 minutes)
    \item Trouver 5 coéquipiers via Discord/forums (30-60 minutes)
    \item Coordonner les disponibilités via messages (15 minutes)
    \item Résultat : 1h30 à 2h de préparation pour 2h de jeu effectif
\end{enumerate}

\textbf{Problématique métier :} 
\textit{La fragmentation des outils d'organisation des raids Destiny 2 génère une perte de productivité estimée à 45 minutes par session et un taux d'abandon de 40\% chez les nouveaux joueurs, impactant négativement l'expérience globale et la rétention des joueurs.}

\textbf{Objectifs SMART :}
\begin{itemize}
    \item \textbf{Spécifique :} Développer une plateforme unifiée intégrant système d'escouades, calendrier collaboratif et guides interactifs pour 3 raids principaux
    \item \textbf{Mesurable :} 
    \begin{itemize}
        \item Réduction du temps d'organisation de 45 à 20 minutes par session (-55\%)
        \item Diminution du taux d'abandon des nouveaux joueurs de 40\% à 25\%
        \item Atteinte de 500 utilisateurs actifs mensuels
    \end{itemize}
    \item \textbf{Atteignable :} MVP livrable en 8 mois avec stack technique maîtrisée (React/Node.js/PostgreSQL) et ressources disponibles
    \item \textbf{Pertinent :} Alignement démontré avec les besoins de la communauté (enquête préalable auprès de 50 joueurs montrant 85\% d'intérêt)
    \item \textbf{Temporel :} 
    \begin{itemize}
        \item Déploiement MVP : 15 mars 2026
        \item Version complète : 15 mai 2026
        \item Mesure des indicateurs : 30 juin 2026
    \end{itemize}
\end{itemize}

\textbf{Impact métier attendu :}
\begin{itemize}
    \item \textbf{Gain de temps :} 25 minutes économisées par session × 4 sessions/mois × 500 joueurs = 833 heures mensuelles gagnées
    \item \textbf{Meilleure rétention :} 15\% de joueurs supplémentaires complétant leur premier raid
    \item \textbf{Expérience unifiée :} Réduction de 70\% du nombre d'applications utilisées simultanément
    \item \textbf{Engagement communautaire :} Augmentation de 25\% du temps de jeu sur les activités complexes
\end{itemize}

\textbf{Indicateurs de succès quantifiés :}
\begin{itemize}
    \item \textbf{Taux d'adoption :} 500 utilisateurs actifs mensuels d'ici juin 2026
    \item \textbf{Satisfaction utilisateur :} Note moyenne ≥ 4.2/5 sur Store Web
    \item \textbf{Gain de temps :} Réduction mesurée du temps d'organisation à ≤ 20 minutes
    \item \textbf{Performance technique :} Temps de réponse API < 500ms pour 95\% des requêtes
    \item \textbf{Fiabilité :} Disponibilité ≥ 99\% sur période de 3 mois
\end{itemize}

\textbf{Diagramme de contexte :}
\begin{verbatim}
                       +===================================+
                       |      DESTINY RAID COMPANION       |
                       |  Plateforme d'organisation raids  |
                       +===================================+
                          |              |              |
                +---------------+    +---------+    +----------+
                |               |    |         |    |          |
        +--------+-------+  +----+------+ +------+----+  +--------+------+
        |  UTILISATEURS  |  | LEADERS   | |  ADMIN    |  |  API EXTERNE  |
        |                |  |           | |           |  |               |
        | • Joueurs      |  | • Création| | • Gestion |  | • Bungie API  |
        | • Consultation |  |  escouades| |  contenu  |  | • Données jeu |
        | • Participation|  | • Planning| | • Stats   |  | • Profils     |
        +----------------+  +-----------+ +-----------+  +---------------+
                ↑                 ↑               ↑               ↑
                |                 |               |               |
        +-------+-------+ +-------+------+ +------+-------+ +-----+-------+
        |   Frontend    | |  Logique     | |  Backend     | |  Services   |
        |   React       | |  Métier      | |  API Node.js | |  Externes   |
        |   Interface   | |  Gestion     | |  Contrôleurs | |  Cache      |
        |   Utilisateur | |  règles      | |  Validation  | |  Données    |
        +---------------+ +--------------+ +--------------+ +-------------+
                                   |               |
                          +--------+---------------+--------+
                          |         PERSISTANCE             |
                          |                                 |
                          |  +--------------------------+   |
                          |  |      PostgreSQL          |   |
                          |  | • Utilisateurs           |   |
                          |  | • Escouades              |   |
                          |  | • Calendrier             |   |
                          |  | • Statistiques           |   |
                          |  +--------------------------+   |
                          |                                 |
                          +---------------------------------+
\end{verbatim}

\textbf{Flux principaux :}
\begin{itemize}
    \item \textbf{Flux utilisateur :} Authentification → Consultation guide → Création/recherche escouade → Planification session
    \item \textbf{Flux données :} API Bungie → Cache Redis → Base PostgreSQL → Frontend React
    \item \textbf{Flux administration :} Authentification admin → Gestion contenu → Analytics → Modération
\end{itemize}

\begin{conseil}
\textbf{Ce que le jury attend dans cette section :}
\begin{itemize}
    \item Une problématique clairement identifiée et justifiée
    \item Des objectifs SMART précis et mesurables
    \item Une compréhension des enjeux métier
    \item Des indicateurs de succès quantifiés
    \item Un diagramme de contexte montrant les acteurs
\end{itemize}

\textbf{Conseils de rédaction :}
\begin{itemize}
    \item Soyez spécifique sur les impacts négatifs actuels
    \item Quantifiez vos objectifs (pourcentages, délais, volumes)
    \item Montrez la pertinence métier de votre projet
    \item Utilisez des diagrammes pour clarifier les interactions
\end{itemize}
\end{conseil}

\begin{jury}
\textbf{Questions de contrôle du jury :}
\begin{itemize}
    \item Pouvez-vous expliquer clairement la problématique que votre projet résout ?
    \item Vos objectifs sont-ils vraiment SMART (spécifiques, mesurables, atteignables, pertinents, temporels) ?
    \item Comment mesurez-vous le succès de votre projet ?
    \item Quels sont les bénéfices attendus pour l'entreprise ?
    \item Pouvez-vous présenter un diagramme de contexte de votre projet ?
    \item Quelles sont les contraintes temporelles et budgétaires ?
\end{itemize}
\end{jury}

\section{Liens utiles}

\begin{itemize}
    \item GitHub Project: \url{https://github.com/votre-username/destiny-raid-companion}
    \item SMART Goals: \url{https://bit.ly/smart-goals-atlassian}
    \item Project Management Institute: \url{https://www.pmi.org/}
    \item Agile Manifesto: \url{https://agilemanifesto.org/}
    \item Business Model Canvas: \url{https://bit.ly/business-model-canvas}
\end{itemize}