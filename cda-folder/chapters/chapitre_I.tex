\chapter{Présentation personnelle et du projet}

\section{Rôle du candidat et contexte}

Dans cette section, vous devez présenter votre rôle dans le projet et le contexte organisationnel. Le jury attend une explication claire de vos responsabilités et de l'environnement de travail dans lequel s'inscrit votre projet CDA.

\textbf{Votre rôle :} \textit{Concepteur et développeur fullstack en autonomie totale}

\textbf{Contexte organisationnel :} \textit{Le projet Destiny Raid Companion s'inscrit dans un contexte de développement indépendant, visant à répondre aux besoins spécifiques de la communauté des joueurs de Destiny 2. Les enjeux métier consistent à améliorer l'expérience utilisateur dans les activités complexes du jeu (raids et donjons) en offrant une plateforme centralisée. Les contraintes techniques incluent l'intégration de l'API Bungie, le développement d'une architecture scalable et la gestion des données temps réel. Les contraintes organisationnelles concernent le travail en autonomie avec une planification rigoureuse sur 8 mois.}

\textbf{Durée et planning :} \textit{Le projet s'étend sur une période de \textbf{huit mois}, d'octobre 2025 à mai 2026, à raison de 2 jours par semaine (environ 60 à 70 jours effectifs).
Les grandes phases sont :
Octobre : cadrage du projet, installation de l'environnement, maquettes.
Novembre – Décembre : développement backend (API, base de données, authentification Bungie).
Janvier – Février : développement frontend (guides, escouades, calendrier, profils).
Mars : intégration de l'API Destiny 2.
Avril : phase de tests unitaires.
Mai : dockerisation, CI/CD et déploiement.}

\begin{exemple}
\textbf{Exemple de pitch QQOQCP :}
\begin{itemize}
    \item \textbf{Quoi :} Application de gestion des projets internes
    \item \textbf{Qui :} Équipes de développement et management
    \item \textbf{Où :} Environnement cloud hybride
    \item \textbf{Quand :} Déploiement progressif sur 6 mois
    \item \textbf{Comment :} Architecture microservices avec React/Node.js
    \item \textbf{Pourquoi :} Centraliser et optimiser le suivi des projets
\end{itemize}

\textbf{Votre pitch QQOQCP :} \textit{
\begin{itemize}
    \item \textbf{Quoi :} Destiny Raid Companion - plateforme web/mobile avec guides interactifs, gestion d'escouades, calendrier collaboratif et recommandations d'équipements
    \item \textbf{Qui :} Joueurs de Destiny 2 (débutants et expérimentés), développé en autonomie complète
    \item \textbf{Où :} Application web responsive déployée sur cloud
    \item \textbf{Quand :} Développement octobre 2025 - mai 2026, déploiement avant soutenance
    \item \textbf{Comment :} Architecture 3-tiers (React/Node.js/PostgreSQL) avec API Bungie et MongoDB
    \item \textbf{Pourquoi :} Résoudre les problèmes d'organisation et de compréhension des raids/donjons
\end{itemize}
}
\end{exemple}

\begin{conseil}
\textbf{Ce que le jury attend dans cette section :}
\begin{itemize}
    \item Une présentation claire de votre rôle et de vos responsabilités
    \item Une explication du contexte métier et des enjeux
    \item Une justification de la pertinence du projet
    \item Un pitch QQOQCP synthétique et percutant
    \item Des indicateurs de succès mesurables
\end{itemize}

\textbf{Conseils de rédaction :}
\begin{itemize}
    \item Soyez précis sur votre fonction (évitez les généralités)
    \item Montrez votre compréhension des enjeux métier
    \item Justifiez le choix de votre projet
    \item Utilisez des chiffres et des métriques quand c'est possible
\end{itemize}
\end{conseil}

\begin{jury}
\begin{itemize}
    \item Pouvez-vous présenter votre rôle précis dans ce projet ?
    \item Quel est le contexte métier de votre entreprise ?
    \item Quels sont les enjeux techniques principaux ?
    \item Comment mesurez-vous le succès de votre projet ?
    \item Quelles sont les contraintes temporelles et budgétaires ?
\end{itemize}
\end{jury}

\section{Problématique et objectifs SMART}

Dans cette section, vous devez identifier clairement la problématique que votre projet résout et définir des objectifs SMART mesurables. Le jury attend une analyse précise des enjeux et des bénéfices attendus.

\textbf{Problématique identifiée :} \textit{Les raids et donjons de Destiny 2 manquent de clarté et de soutien intégré : les joueurs peinent à comprendre les mécaniques, à trouver des coéquipiers fiables et à planifier leurs sessions, ce qui entraîne frustration et abandon des activités complexes.}

\textbf{Objectifs SMART :} \textit{
\begin{itemize}
    \item \textbf{Spécifique :} Développer une plateforme centralisée avec guides interactifs, gestion d'escouades et calendrier collaboratif
    \item \textbf{Mesurable :} Réduire de 50\% le temps de recherche d'équipe et de préparation des raids
    \item \textbf{Atteignable :} Livrer la version MVP en 8 mois avec les technologies maîtrisées (React/Node.js)
    \item \textbf{Pertinent :} Améliorer l'expérience globale des joueurs et favoriser l'engagement communautaire
    \item \textbf{Temporel :} Déploiement complet avant la soutenance CDA en mai 2026
\end{itemize}}

\textbf{Bénéfices attendus :} \textit{
\begin{itemize}
    \item Centralisation des outils dispersés (guides, LFG, planning)
    \item Amélioration de l'accessibilité pour les nouveaux joueurs
    \item Optimisation du temps d'organisation des sessions de jeu
    \item Renforcement de l'engagement communautaire via la gamification
\end{itemize}}

\begin{exemple}
\textbf{Exemple d'objectifs SMART :}
\begin{itemize}
    \item \textbf{Spécifique :} Développer une plateforme unifiée de gestion de projet
    \item \textbf{Mesurable :} Réduire de 40\% le temps de reporting hebdomadaire
    \item \textbf{Atteignable :} Livrer la v1 en 6 mois avec l'équipe actuelle
    \item \textbf{Pertinent :} Aligner les outils sur la stratégie digitale
    \item \textbf{Temporel :} Déploiement complet avant fin d'année 2025
\end{itemize}

\textbf{Vos objectifs SMART :} \textit{
\begin{itemize}
    \item \textbf{Spécifique :} Développer une plateforme centralisée avec guides interactifs, gestion d'escouades et calendrier collaboratif
    \item \textbf{Mesurable :} Réduire de 50\% le temps de recherche d'équipe et de préparation des raids
    \item \textbf{Atteignable :} Livrer la version MVP en 8 mois avec les technologies maîtrisées (React/Node.js)
    \item \textbf{Pertinent :} Améliorer l'expérience globale des joueurs et favoriser l'engagement communautaire
    \item \textbf{Temporel :} Déploiement complet avant la soutenance CDA en mai 2026
\end{itemize}}

\textbf{Votre diagramme de contexte :}
\begin{verbatim}
                    +=================================+
                    |     DESTINY RAID COMPANION      |
                    |                                 |
                    +=================================+
                                      |
                    +-----------------+-----------------+
                    |                 |                 |
            +-------+-------+   +-------+-------+   +-------+--------+
            | API Bungie:   |   | Base de       |   | Joueurs:       |
            | données       |   | données:      |   | débutants      |
            | raids/équip   |   | utilisateurs, |   | et expérimentés|
            +---------------+   | escouades     |   +----------------+
                    |           +---------------+         |
                    +-----------------+-----------------+
                                      |
                              +-------+-------+
                              | Cloud:        |
                              | déploiement   |
                              +---------------+
\end{verbatim}
\end{exemple}

\begin{conseil}
\textbf{Ce que le jury attend dans cette section :}
\begin{itemize}
    \item Une problématique clairement identifiée et justifiée
    \item Des objectifs SMART précis et mesurables
    \item Une compréhension des enjeux métier
    \item Des indicateurs de succès quantifiés
    \item Un diagramme de contexte montrant les acteurs
\end{itemize}

\textbf{Conseils de rédaction :}
\begin{itemize}
    \item Soyez spécifique sur les impacts négatifs actuels
    \item Quantifiez vos objectifs (pourcentages, délais, volumes)
    \item Montrez la pertinence métier de votre projet
    \item Utilisez des diagrammes pour clarifier les interactions
\end{itemize}
\end{conseil}

\begin{jury}
\textbf{Questions de contrôle du jury :}
\begin{itemize}
    \item Pouvez-vous expliquer clairement la problématique que votre projet résout ?
    \item Vos objectifs sont-ils vraiment SMART (spécifiques, mesurables, atteignables, pertinents, temporels) ?
    \item Comment mesurez-vous le succès de votre projet ?
    \item Quels sont les bénéfices attendus pour l'entreprise ?
    \item Pouvez-vous présenter un diagramme de contexte de votre projet ?
    \item Quelles sont les contraintes temporelles et budgétaires ?
\end{itemize}
\end{jury}

\section{Liens utiles}

\begin{itemize}
    \item GitHub About: \url{https://docs.github.com/}
    \item SMART Goals: \url{https://bit.ly/smart-goals-atlassian}
    \item Project Management Institute: \url{https://www.pmi.org/}
    \item Agile Manifesto: \url{https://agilemanifesto.org/}
    \item Business Model Canvas: \url{https://bit.ly/business-model-canvas}
\end{itemize}
