\chapter{Présentation personnelle et du projet}

\section{Rôle du candidat et contexte}

\textbf{Mon rôle :} \textit{Concepteur et développeur fullstack en autonomie totale - Responsable de la conception technique, du développement, des tests et du déploiement de la plateforme Destiny Raid Companion.
\newline}

\textbf{Contexte organisationnel :} \textit{Étant un joueur vétéran du jeu Destiny 2 j'ai remarquer plusieurs problèmatique présentes. La problèmatique principale est le fait qu'il y a des missions complexe avec aucune explication sur comment les réaliser. J'ai aussi vu que sur 10 équipe différentes la méthode ne viens pas de la même source donc le fonctionnement change. De plus pour ce qui concerne les nouveaux joueurs et les joueurs qui n'ont pas jouer depuis longtemps alors ils auront un problème pour rejoindre des équipes car ils ne connaissent pas les méchaniques et la majorité des autres joueurs ne cherche seulement des perssonnes qui savent quoi faire. En remarquant cela j'ai eu l'idée de mettre en place une plateforme web qui pourrat résoudre ce problème récurent pour les joueurs. Le projet Destiny Raid Companion répond à un double besoin métier : optimiser l'organisation des activités complexes pour les joueurs expérimentés ET résoudre les barrières d'entrée pour les nouveaux joueurs confrontés à la complexité des raids Destiny 2.}
\newline

\textbf{Processus métier concernés :}
\begin{itemize}
    \item \textbf{Planification des sessions :} Actuellement via Discord + Google Calendar -> Processus non standardisé
    \item \textbf{Apprentissage des mécaniques :} Consultation de guides sur 3-4 sites différents -> Information dispersée et incohérente
    \item \textbf{Recrutement d'équipe :} Utilisation de forums et LFG (Looking for Group) -> Matching non optimisé, surtout pour débutants
    \item \textbf{Suivi de progression :} Notes manuelles ou tableurs Excel -> Données non centralisées
    \item \textbf{Onboarding nouveaux joueurs :} Processus informel dépendant de la bienveillance des joueurs expérimentés
    \newline
\end{itemize}

\textbf{Durée et planning :} \textit{Le projet s'étend sur une période de \textbf{huit mois}, d'octobre 2025 à mai 2026, à raison de 2 jours par semaine (environ 60 à 70 jours effectifs).
Les grandes phases sont :
\begin{itemize}
    \item \textbf{Octobre :} Cadrage du projet, installation de l'environnement, maquettes
    \item \textbf{Novembre – Décembre :} Développement backend (API, base de données, authentification Bungie)
    \item \textbf{Janvier – Février :} Développement frontend (guides, escouades, calendrier, profils)
    \item \textbf{Mars :} Intégration de l'API Destiny 2
    \item \textbf{Avril :} Phase de tests unitaires et validation utilisateur
    \item \textbf{Mai :} Dockerisation, CI/CD et déploiement production
    \newline
\end{itemize}}

\textbf{Votre pitch QQOQCP :}
\begin{itemize}
    \item \textbf{Quoi :} Destiny Raid Companion - plateforme web centralisant guides interactifs, gestion d'escouades et calendrier collaboratif
    \item \textbf{Qui :} Joueurs de Destiny 2 (débutants cherchant de la clarté et joueurs expérimentés recherchant l'optimisation)
    \item \textbf{Où :} Application web responsive accessible sur tous devices, déployée sur cloud
    \item \textbf{Quand :} Développement octobre 2025 - mai 2026, MVP déployé en mars 2026
    \item \textbf{Comment :} Architecture 3-tiers (React/Node.js/PostgreSQL) avec intégration API Bungie
    \item \textbf{Pourquoi :} Réduire de 50\% le temps d'organisation et faciliter l'accès aux raids pour les nouveaux joueurs
\end{itemize}

\begin{conseil}
\textbf{Ce que le jury attend dans cette section :}
\begin{itemize}
    \item Une présentation claire de votre rôle et de vos responsabilités
    \item Une explication du contexte métier et des enjeux
    \item Une justification de la pertinence du projet
    \item Un pitch QQOQCP synthétique et percutant
    \item Des indicateurs de succès mesurables
\end{itemize}

\textbf{Conseils de rédaction :}
\begin{itemize}
    \item Soyez précis sur votre fonction (évitez les généralités)
    \item Montrez votre compréhension des enjeux métier
    \item Justifiez le choix de votre projet
    \item Utilisez des chiffres et des métriques quand c'est possible
\end{itemize}
\end{conseil}

\begin{jury}
\begin{itemize}
    \item Pouvez-vous présenter votre rôle précis dans ce projet ?
    \item Quel est le contexte métier de votre entreprise ?
    \item Quels sont les enjeux techniques principaux ?
    \item Comment mesurez-vous le succès de votre projet ?
    \item Quelles sont les contraintes temporelles et budgétaires ?
\end{itemize}
\end{jury}

\section{Problématique et objectifs SMART}

\textbf{Problématique métier globale :} 
\textit{La fragmentation des outils d'organisation et le manque de clarté des informations génèrent une perte de productivité estimée à 45 minutes par session pour les expérimentés et un taux d'abandon de 40\% chez les nouveaux joueurs, impactant négativement l'expérience globale et la rétention.
\newline}

\textbf{Problématique nouveaux joueurs :}
\begin{itemize}
    \item \textbf{Manque de clarté :} Mécaniques de raids complexes sans guide intégré au jeu
    \item \textbf{Information dispersée :} Guides éparpillés sur YouTube, Reddit, sites spécialisés
    \item \textbf{Barrière sociale :} Difficulté à trouver des équipes acceptant des débutants
    \item \textbf{Peur de l'échec :} Appréhension de "gâcher" l'expérience des joueurs expérimentés
    \newline
\end{itemize}

\textbf{Cas d'usage concret - Nouveau joueur :}
\textit{Thomas, 25 ans, souhaite réaliser son premier raid "Vault of Glass" mais :}
\begin{enumerate}
    \item \textbf{Recherche d'information :} Consulte 3-4 sites différents + vidéos YouTube (45-60 minutes)
    \item \textbf{Incompréhension :} Mécaniques complexes mal expliquées, termes techniques non définis
    \item \textbf{Difficulté recrutement :} Refusé par 5 équipes pour "manque d'expérience"
    \item \textbf{Perte de motivation :} Abandon après 2 heures de tentatives infructueuses
    \newline
\end{enumerate}

\textbf{Cas d'usage concret - Joueur expérimenté :}
\textit{Sarah, 30 ans, leader de clan, organise des raids hebdomadaires mais :}
\begin{enumerate}
    \item \textbf{Coordination complexe :} Messages Discord, appels vocaux, vérification disponibilités (30 minutes)
    \item \textbf{Formation débutants :} Doit répéter les explications à chaque nouvelle recrue
    \item \textbf{Suivi difficile :} Progression non centralisée, oublis fréquents
    \newline
\end{enumerate}

\textbf{Objectifs SMART :}
\begin{itemize}
    \item \textbf{Spécifique :} Développer une plateforme unifiée avec guides interactifs clarifiés, système d'escouades inclusif et calendrier collaboratif
    \item \textbf{Mesurable :} 
    \begin{itemize}
        \item Réduction du temps d'organisation de 45 à 20 minutes par session (-55\%)
        \item Diminution du taux d'abandon des nouveaux joueurs de 40\% à 20\%
        \item Réduction du temps d'apprentissage des mécaniques de 60 à 25 minutes
        \item Atteinte de 500 utilisateurs actifs mensuels
    \end{itemize}
    \item \textbf{Atteignable :} MVP livrable en 8 mois avec stack technique maîtrisée (React/Node.js/PostgreSQL) et ressources disponibles
    \item \textbf{Pertinent :} Alignement démontré avec les besoins des deux segments (enquête préalable montrant 85\% d'intérêt chez les débutants et 70\% chez les expérimentés)
    \item \textbf{Temporel :} 
    \begin{itemize}
        \item Déploiement MVP : 15 mars 2026
        \item Version complète : 15 mai 2026
        \item Mesure des indicateurs : 30 juin 2026
        \newline
    \end{itemize}
\end{itemize}

\textbf{Impact métier attendu :}
\begin{itemize}
    \item \textbf{Pour les nouveaux joueurs :}
    \begin{itemize}
        \item Accès simplifié aux informations claires et structurées
        \item Matching avec équipes acceptant les débutants
        \item Réduction de la courbe d'apprentissage
        \newline
    \end{itemize}
    \item \textbf{Pour les joueurs expérimentés :}
    \begin{itemize}
        \item Gain de temps sur l'organisation : 25 minutes/session
        \item Centralisation des outils : fin de la dispersion
        \item Meilleure gestion des équipes et de la progression
        \newline
    \end{itemize}
    \item \textbf{Impact communautaire :}
    \begin{itemize}
        \item 833 heures mensuelles gagnées (calcul : 25 min × 4 sessions × 500 joueurs)
        \item 15\% de joueurs supplémentaires complétant leur premier raid
        \item Augmentation de 25\% du temps de jeu sur les activités complexes
        \newline
    \end{itemize}
\end{itemize}

\textbf{Indicateurs de succès quantifiés :}
\begin{itemize}
    \item \textbf{Taux d'adoption :} 500 utilisateurs actifs mensuels d'ici juin 2026
    \item \textbf{Satisfaction nouveaux joueurs :} Note moyenne ≥ 4.5/5 sur la clarté des guides
    \item \textbf{Gain de temps :} Réduction mesurée du temps d'organisation à ≤ 20 minutes
    \item \textbf{Rétention débutants :} Taux d'abandon premier raid réduit à ≤ 20\%
    \item \textbf{Performance technique :} Temps de réponse API < 500ms pour 95\% des requêtes
    \newline
\end{itemize}

\textbf{Diagramme de contexte :}

\begin{figure}
    \centering
    \includegraphics[height=5cm]{assets/diagramme.png}
    \caption{Diagramme de contexte de la plateforme Destiny Raid Companion}
    \label{fig:contexte}
\end{figure}

\textbf{Flux principaux :}
\begin{itemize}
    \item \textbf{Flux nouveau joueur :} Authentification -> Guide débutant -> Recherche équipe bienveillante -> Session d'apprentissage
    \item \textbf{Flux joueur expérimenté :} Authentification -> Création escouade -> Planification rapide -> Gestion équipe
    \item \textbf{Flux données :} API Bungie -> Cache Redis -> Base PostgreSQL -> Frontend React
    \item \textbf{Flux matching :} Algorithmes de matching débutants/expérimentés -> Suggestions d'équipes
    \newline
\end{itemize}

\begin{conseil}
\textbf{Ce que le jury attend dans cette section :}
\begin{itemize}
    \item Une problématique clairement identifiée et justifiée
    \item Des objectifs SMART précis et mesurables
    \item Une compréhension des enjeux métier
    \item Des indicateurs de succès quantifiés
    \item Un diagramme de contexte montrant les acteurs
\end{itemize}

\textbf{Conseils de rédaction :}
\begin{itemize}
    \item Soyez spécifique sur les impacts négatifs actuels
    \item Quantifiez vos objectifs (pourcentages, délais, volumes)
    \item Montrez la pertinence métier de votre projet
    \item Utilisez des diagrammes pour clarifier les interactions
\end{itemize}
\end{conseil}

\begin{jury}
\textbf{Questions de contrôle du jury :}
\begin{itemize}
    \item Pouvez-vous expliquer clairement la problématique que votre projet résout ?
    \item Vos objectifs sont-ils vraiment SMART (spécifiques, mesurables, atteignables, pertinents, temporels) ?
    \item Comment mesurez-vous le succès de votre projet ?
    \item Quels sont les bénéfices attendus pour l'entreprise ?
    \item Pouvez-vous présenter un diagramme de contexte de votre projet ?
    \item Quelles sont les contraintes temporelles et budgétaires ?
\end{itemize}
\end{jury}

\section{Liens utiles}

\begin{itemize}
    \item GitHub Project: \url{https://github.com/votre-username/destiny-raid-companion}
    \item SMART Goals: \url{https://bit.ly/smart-goals-atlassian}
    \item Project Management Institute: \url{https://www.pmi.org/}
    \item Agile Manifesto: \url{https://agilemanifesto.org/}
    \item Business Model Canvas: \url{https://bit.ly/business-model-canvas}
\end{itemize}