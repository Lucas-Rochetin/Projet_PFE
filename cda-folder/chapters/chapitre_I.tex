\chapter{Présentation personnelle et du projet}

\section{Rôle du candidat et contexte}

Dans cette section, vous devez présenter votre rôle dans le projet et le contexte organisationnel. Le jury attend une explication claire de vos responsabilités et de l'environnement de travail dans lequel s'inscrit votre projet CDA.
\newline
\textbf{Votre rôle :} \textit{Concepteur et développeur fullstack en autonomie totale}
\newline

\textbf{Contexte organisationnel :} \textit{Le projet Destiny Raid Companion répond à un besoin concret identifié dans la communauté Destiny 2 : la difficulté d'organisation des activités complexes (raids et donjons) qui nécessitent coordination et préparation. Actuellement, les joueurs utilisent des outils dispersés (Discord pour la communication, sites externes pour les guides, applications mobiles pour le suivi) ce qui entraîne une perte de temps estimée à 30 minutes par session et un taux d'abandon de 40\% chez les nouveaux joueurs.}
\newline
\textbf{Durée et planning :} \textit{Le projet s'étend sur une période de \textbf{huit mois}, d'octobre 2025 à mai 2026, à raison de 2 jours par semaine (environ 60 à 70 jours effectifs).
Les grandes phases sont :
Octobre : cadrage du projet, installation de l'environnement, maquettes.
Novembre – Décembre : développement backend (API, base de données, authentification Bungie).
Janvier – Février : développement frontend (guides, escouades, calendrier, profils).
Mars : intégration de l'API Destiny 2.
Avril : phase de tests unitaires.
Mai : dockerisation, CI/CD et déploiement.}
\newline


\textbf{Votre pitch QQOQCP :} \textit{
\begin{itemize}
    \item \textbf{Quoi :} Destiny Raid Companion - plateforme web centralisant guides interactifs, gestion d'escouades et calendrier collaboratif
    \item \textbf{Qui :} Joueurs de Destiny 2 (cibles : débutants frustrés et joueurs expérimentés recherchant l'optimisation)
    \item \textbf{Où :} Application web responsive accessible sur tous devices
    \item \textbf{Quand :} Développement octobre 2025 - mai 2026, MVP déployé en mars 2026
    \item \textbf{Comment :} Architecture 3-tiers (React/Node.js/PostgreSQL) avec intégration API Bungie
    \item \textbf{Pourquoi :} Réduire de 50\% le temps d'organisation et diminuer de 40\% l'abandon des raids par les nouveaux joueurs
\newline
\end{itemize}
}


\begin{conseil}
\textbf{Ce que le jury attend dans cette section :}
\begin{itemize}
    \item Une présentation claire de votre rôle et de vos responsabilités
    \item Une explication du contexte métier et des enjeux
    \item Une justification de la pertinence du projet
    \item Un pitch QQOQCP synthétique et percutant
    \item Des indicateurs de succès mesurables
    \newline
\end{itemize}

\textbf{Conseils de rédaction :}
\begin{itemize}
    \item Soyez précis sur votre fonction (évitez les généralités)
    \item Montrez votre compréhension des enjeux métier
    \item Justifiez le choix de votre projet
    \item Utilisez des chiffres et des métriques quand c'est possible
    \newline
\end{itemize}
\end{conseil}

\begin{jury}
\begin{itemize}
    \item Pouvez-vous présenter votre rôle précis dans ce projet ?
    \item Quel est le contexte métier de votre entreprise ?
    \item Quels sont les enjeux techniques principaux ?
    \item Comment mesurez-vous le succès de votre projet ?
    \item Quelles sont les contraintes temporelles et budgétaires ?
    \newline
\end{itemize}
\end{jury}

\section{Problématique et objectifs SMART}

Dans cette section, vous devez identifier clairement la problématique que votre projet résout et définir des objectifs SMART mesurables. Le jury attend une analyse précise des enjeux et des bénéfices attendus.
\newline
\newline
\textbf{Problématique identifiée :} \textit{Les joueurs de Destiny 2 perdent en moyenne 30 minutes par session à organiser leurs raids via des outils dispersés (Discord, sites de guides séparés, applications de planning), ce qui entraîne une frustration notable et un taux d'abandon de 40\% chez les nouveaux joueurs lors de leur première expérience de raid.}
\newline
\newline
\textbf{Objectifs SMART :} \textit{
\begin{itemize}
    \item \textbf{Spécifique :} Développer une plateforme unifiée avec système d'escouades, calendrier intégré et guides interactifs
    \item \textbf{Mesurable :} Réduire le temps d'organisation de 50\% (de 30 à 15 minutes) et diminuer l'abandon des nouveaux joueurs de 40\% à 20\%
    \item \textbf{Atteignable :} MVP livrable en 8 mois avec stack technique maîtrisée (React/Node.js/PostgreSQL)
    \item \textbf{Pertinent :} Alignement total avec les besoins de la communauté Destiny 2 mesurés par enquête préalable
    \item \textbf{Temporel :} Déploiement du MVP pour mars 2026 et version complète pour mai 2026 (soutenance)
    \newline
\end{itemize}}

\textbf{Bénéfices attendus :} \textit{
\begin{itemize}
    \item Gain de temps : 15 minutes économisées par session de raid
    \item Meilleure rétention : réduction de 20\% de l'abandon des nouveaux joueurs
    \item Expérience utilisateur unifiée : fin de la dispersion entre multiples applications
    \item Engagement communautaire renforcé via système de gamification
    \newline
\end{itemize}}

\textbf{Indicateurs de succès :}
\begin{itemize}
    \item \textbf{Taux d'adoption :} 100 utilisateurs actifs mensuels d'ici juin 2026
    \textbf{Satisfaction utilisateur :} Note moyenne de 4/5 sur les stores d'application
    \textbf{Performance technique :} Temps de réponse API < 500ms pour 95\% des requêtes
    \textbf{Couverture tests :} > 80\% de couverture de code par les tests automatisés
    \newline
\end{itemize}

\textbf{Diagramme de contexte :}
\begin{verbatim}
                       +===================================+
                       |      DESTINY RAID COMPANION       |
                       |                                   |
                       +===================================+
                          |              |              |
                +---------------+    +---------+    +----------+
                |               |    |         |    |          |
        +--------+-------+  +----+----+  +-------+---+  +--------+-------+
        |  Joueurs       |  | Leaders  |  | Admin     |  | Système       |
        | (Utilisateurs) |  | Escouade |  | (Gestion  |  | Bungie API    |
        |                |  |          |  | contenu)  |  |               |
        +----------------+  +----------+  +-----------+  +---------------+
                |                 |              |               |
                +-----------------+--------------+---------------+
                                         |
                               +---------+---------+
                               |  Base de données  |
                               |  PostgreSQL       |
                               +-------------------+

\end{verbatim}
\begin{conseil}
\textbf{Ce que le jury attend dans cette section :}
\begin{itemize}
    \item Une problématique clairement identifiée et justifiée
    \item Des objectifs SMART précis et mesurables
    \item Une compréhension des enjeux métier
    \item Des indicateurs de succès quantifiés
    \item Un diagramme de contexte montrant les acteurs
\end{itemize}

\textbf{Conseils de rédaction :}
\begin{itemize}
    \item Soyez spécifique sur les impacts négatifs actuels
    \item Quantifiez vos objectifs (pourcentages, délais, volumes)
    \item Montrez la pertinence métier de votre projet
    \item Utilisez des diagrammes pour clarifier les interactions
\end{itemize}
\end{conseil}

\begin{jury}
\textbf{Questions de contrôle du jury :}
\begin{itemize}
    \item Pouvez-vous expliquer clairement la problématique que votre projet résout ?
    \item Vos objectifs sont-ils vraiment SMART (spécifiques, mesurables, atteignables, pertinents, temporels) ?
    \item Comment mesurez-vous le succès de votre projet ?
    \item Quels sont les bénéfices attendus pour l'entreprise ?
    \item Pouvez-vous présenter un diagramme de contexte de votre projet ?
    \item Quelles sont les contraintes temporelles et budgétaires ?
\end{itemize}
\end{jury}

\section{Liens utiles}

\begin{itemize}
    \item GitHub About: \url{https://docs.github.com/}
    \item SMART Goals: \url{https://bit.ly/smart-goals-atlassian}
    \item Project Management Institute: \url{https://www.pmi.org/}
    \item Agile Manifesto: \url{https://agilemanifesto.org/}
    \item Business Model Canvas: \url{https://bit.ly/business-model-canvas}
\end{itemize}