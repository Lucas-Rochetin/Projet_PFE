\chapter{Méthodologie et organisation}

\section{Gestion de projet avec GitHub}

Dans cette section, vous devez présenter votre approche de gestion de projet entièrement centralisée sur GitHub. Le jury attend une démonstration de votre maîtrise des outils GitHub pour la planification, le suivi et la réalisation de votre projet.

\textbf{Votre approche GitHub :} \textit{Le projet Destiny 2 Raid Companion est géré entièrement sur GitHub avec une approche Agile adaptée au développement en solo. L'organisation repose sur GitHub Projects pour le suivi des tâches, les Milestones pour la planification temporelle, et un workflow Git Flow modifié pour assurer la qualité du code.}

\subsection{Adaptation de la méthode Agile au contexte}

\textbf{Pourquoi l'Agile est adapté à ce projet :}
\begin{itemize}
    \item \textbf{Projet innovant :} Besoin de s'adapter aux retours utilisateurs rapidement
    \item \textbf{API tierce complexe :} Nécessité d'itérer sur l'intégration Bungie API
    \item \textbf{Développement solo :} Flexibilité pour ajuster les priorités selon les blocages
    \item \textbf{Validation continue :} MVP à tester rapidement avec la communauté Destiny 2
\end{itemize}

\textbf{Application des principes Agile au quotidien :}
\begin{itemize}
    \item \textbf{Itérations courtes :} Sprints de 2 semaines pour maintenir le rythme
    \item \textbf{Adaptation permanente :} Revue des priorités à chaque sprint planning
    \item \textbf{Livraison continue :} Déploiement automatique sur environnement de test
    \item \textbf{Simplicité :} Focus sur les fonctionnalités à plus haute valeur ajoutée
\end{itemize}

\subsection{Rituels Agile et leur mise en œuvre}

\begin{center}
\begin{tabular}{|p{3cm}|p{4cm}|p{4cm}|}
\hline
\textbf{Rituel} & \textbf{Fréquence} & \textbf{Objectif et mise en œuvre} \\
\hline
\textbf{Daily Standup} & Quotidien (10 min) & Point sur avancement, blocages, objectifs journée. Mise à jour GitHub Projects \\
\hline
\textbf{Sprint Planning} & Tous les 15 jours & Sélection des issues du backlog, estimation, définition objectifs sprint \\
\hline
\textbf{Sprint Review} & Fin de sprint & Démonstration fonctionnalités, validation critères d'acceptation, recueil retours \\
\hline
\textbf{Sprint Retrospective} & Fin de sprint & Amélioration processus, identification points à optimiser, actions correctives \\
\hline
\end{tabular}
\end{center}

\begin{exemple}
\textbf{Schéma des rituels Scrum :}
\begin{verbatim}
Sprint Planning (2h) ---> Daily Standup (10min) ---> Sprint Review (1h)
     |                          |                          |
     |                          |                          |
     v                          v                          v
Backlog Refinement         Développement              Sprint Retro (45min)
     |                          |                          |
     |                          |                          |
     +---------> Sprint (15j) <---------+------------------+
\end{verbatim}

\textbf{Métriques de suivi :}
\begin{itemize}
    \item \textbf{Vélocité :} 8-12 story points par sprint
    \item \textbf{Taux de complétion :} > 85\% des tâches par sprint
    \item \textbf{Bugs ouverts/fermés :} Ratio < 0.5 (2 bugs fermés pour 1 ouvert)
    \item \textbf{Lead time :} < 5 jours pour les issues critiques
\end{itemize}

\textbf{Colonnes du tableau Kanban :}
\begin{itemize}
    \item \textbf{Backlog :} Fonctionnalités à développer (triées par priorité)
    \item \textbf{Sprint Backlog :} Tâches sélectionnées pour le sprint courant
    \item \textbf{To Do :} Tâches prêtes pour le développement
    \item \textbf{In Progress :} Tâches en cours (WIP limit: 2)
    \item \textbf{Review :} Code en attente de validation (tests, revue)
    \item \textbf{Done :} Fonctionnalités livrées et validées
\end{itemize}
\end{exemple}

\subsection{User Stories et estimation de temps}

Dans cette sous-section, vous devez détailler vos user stories avec des estimations de temps réalistes. Chaque user story doit être liée à des commits et des milestones GitHub pour un suivi précis de l'avancement.

\textbf{Vos user stories avec estimations :} \textit{
\begin{longtable}{p{3cm}p{4cm}p{2cm}p{2cm}}
\toprule
\textbf{User Story} & \textbf{Critères d'acceptation} & \textbf{Story Points} & \textbf{Milestone} \\
\midrule
En tant que joueur, je veux m'authentifier avec mon compte Bungie & Connexion OAuth fonctionnelle, récupération du profil, gestion des sessions JWT & 5 & \href{https://github.com/xxx/milestone/1}{MVP v1.0} \\
En tant que joueur, je veux consulter des guides interactifs de raids & Affichage des étapes détaillées, illustrations, stratégies par rôle & 8 & \href{https://github.com/xxx/milestone/1}{MVP v1.0} \\
En tant que joueur, je veux créer et gérer une escouade & Création d'escouade, invitation de membres, gestion des rôles & 5 & \href{https://github.com/xxx/milestone/1}{MVP v1.0} \\
En tant que joueur, je veux planifier une session de raid & Calendrier interactif, création d'événement, notifications & 8 & \href{https://github.com/xxx/milestone/2}{v1.1} \\
En tant que joueur, je veux voir mes statistiques et badges & Profil personnel, historique des raids, système de gamification & 5 & \href{https://github.com/xxx/milestone/2}{v1.1} \\
En tant qu'administrateur, je veux gérer le contenu des guides & Interface d'administration, édition des guides, validation & 8 & \href{https://github.com/xxx/milestone/3}{v1.2} \\
\bottomrule
\end{longtable}}

\textbf{Système d'estimation :}
\begin{itemize}
    \item \textbf{1 point :} Tâche simple (< 1 jour)
    \item \textbf{3 points :} Tâche moyenne (1-2 jours)
    \item \textbf{5 points :} Tâche complexe (3-4 jours)
    \item \textbf{8 points :} Tâche très complexe (> 5 jours)
\end{itemize}

\begin{conseil}
\textbf{Ce que le jury attend dans cette section :}
\begin{itemize}
    \item Une justification claire du choix méthodologique
    \item Une description des rituels et de leur utilité
    \item Une adaptation de la méthode au contexte projet
    \item Des métriques de suivi et d'amélioration continue
    \item Une organisation claire des responsabilités
\end{itemize}

\textbf{Conseils de rédaction :}
\begin{itemize}
    \item Expliquez pourquoi cette méthode convient à votre projet
    \item Montrez comment vous mesurez l'efficacité de votre approche
    \item Décrivez les adaptations nécessaires à votre contexte
    \item Utilisez des diagrammes pour illustrer vos processus
\end{itemize}
\end{conseil}

\begin{jury}
\begin{itemize}
    \item Pourquoi avez-vous choisi cette méthode Agile ?
    \item Comment mesurez-vous l'efficacité de vos rituels ?
    \item Quels sont vos indicateurs de performance ?
    \item Comment gérez-vous les imprévus dans vos sprints ?
    \item Avez-vous adapté la méthode à votre contexte ?
\end{itemize}
\end{jury}

\section{Versioning GitHub et conventions}

Le versioning GitHub suit le modèle Git Flow avec des branches spécialisées pour chaque type de développement. Les conventions de nommage et de commit facilitent la traçabilité et la collaboration. Les Pull Requests permettent la revue de code systématique et la validation des fonctionnalités avant intégration.

Les conventions établies couvrent le nommage des branches, le format des messages de commit, et les templates de Pull Request. Cette standardisation améliore la qualité du code et accélère l'onboarding de nouveaux développeurs.

\subsection{CONTRIBUTING.md et normalisation}

\textbf{Contenu du CONTRIBUTING.md :}
\begin{itemize}
    \item \textbf{Environnement :} Setup du projet, pré-requis, installation
    \item \textbf{Conventions de code :} ESLint, Prettier, standards React/Node.js
    \item \textbf{Workflow Git :} Processus de création de branches, commits, PR
    \item \textbf{Testing :} Comment exécuter les tests, couverture attendue
    \item \textbf{Code Review :} Checklist pour la revue de code
\end{itemize}

\subsection{Conventions de branches}

\begin{center}
\begin{tabular}{|p{3cm}|p{8cm}|}
\hline
\textbf{Type de branche} & \textbf{Convention de nommage} \\
\hline
Feature & \texttt{feature/nom-fonctionnalite} ou \texttt{feature/issue-\#123} \\
Bugfix & \texttt{fix/description-bug} ou \texttt{fix/issue-\#456} \\
Hotfix & \texttt{hotfix/description-urgente} \\
Release & \texttt{release/v1.2.3} \\
Documentation & \texttt{docs/sujet-documentation} \\
\hline
\end{tabular}
\end{center}

\subsection{Conventions de commits}

\begin{exemple}
\textbf{Schéma Git Flow :}
\begin{verbatim}
main (protected) -------------------------------
  |                      |          |
  |                      |          +-- hotfix/critical-issue
  |                      |
  +-- develop (protected) ----------------------
       |                    |          |
       |                    |          +-- release/v1.2.0
       |                    |
       +-- feature/user-auth
       +-- feature/raid-guides
       +-- fix/login-validation
\end{verbatim}

\textbf{Conventions de commit (Conventional Commits) :}
\begin{lstlisting}
feat: add OAuth Bungie authentication system
fix: resolve login token expiration issue
docs: update API integration guide
test: add unit tests for user service
refactor: improve raid guide component structure
style: format code with prettier
chore: update dependencies to latest versions
\end{lstlisting}

\textbf{Template de Pull Request :}
\begin{verbatim}
## Description
[Description des changements apportés]

## Issue liée
[Lien vers l'issue GitHub]

## Type de changement
- [ ] Correction de bug
- [ ] Nouvelle fonctionnalité
- [ ] Modification de configuration
- [ ] Documentation

## Checklist
- [ ] Mon code suit les conventions du projet
- [ ] J'ai ajouté des tests
- [ ] Les tests passent localement
- [ ] J'ai mis à jour la documentation
\end{verbatim}
\end{exemple}

\begin{conseil}
\begin{itemize}
    \item Définir des conventions de nommage claires
    \item Utiliser des messages de commit descriptifs
    \item Mettre en place des templates de Pull Request
    \item Configurer des règles de protection des branches
    \item Documenter les conventions dans un CONTRIBUTING.md
\end{itemize}
\end{conseil}

\begin{jury}
\begin{itemize}
    \item Quelles sont vos conventions de versioning ?
    \item Comment gérez-vous les conflits de merge ?
    \item Vos Pull Requests sont-elles systématiquement revues ?
    \item Comment assurez-vous la qualité du code ?
    \item Avez-vous des règles de protection des branches ?
\end{itemize}
\end{jury}

\section{Planification et outils de suivi}

La planification combine une roadmap GitHub pour la vision macro et GitHub Projects pour le suivi opérationnel. La roadmap GitHub visualise les dépendances et les jalons critiques, tandis que le Kanban GitHub Projects offre une vue détaillée des tâches en cours. Cette approche dual optimise la coordination entre la planification stratégique et l'exécution tactique.

La roadmap GitHub permet de communiquer la vision produit et les priorités à long terme. Les milestones et les dépendances facilitent la coordination entre les différentes équipes et la gestion des risques de planning.

\subsection{GitHub Project et Roadmap}

\textbf{Structure du GitHub Project :}
\begin{itemize}
    \item \textbf{Vue Kanban :} Suivi visuel de l'état des tâches
    \item \textbf{Filtres :} Par label, milestone, assigné, statut
    \item \textbf{Automatisations :} Changement de statut basé sur les PR/issues
    \item \textbf{Vues personnalisées :} Tableau de bord pour daily standup
\end{itemize}

\textbf{Roadmap GitHub :}
\begin{itemize}
    \item \textbf{Visibilité :} Roadmap publique pour transparence
    \item \textbf{Milestones :} Dates cibles pour chaque version
    \item \textbf{Dépendances :} Liens entre les fonctionnalités
    \item \textbf{Suivi progression :} Avancement visuel par milestone
\end{itemize}

\begin{exemple}
\textbf{Extrait de roadmap GitHub :}
\begin{verbatim}
Phase 1: MVP (Oct 2025 - Mai 2026)
+-- Sprint 1: Setup & Auth (4 semaines)
    +-- Environnement dev (1 semaine) [Issue #1]
    +-- Authentification Bungie (2 semaines) [Issue #2]
    +-- Base de données (1 semaine) [Issue #3]
    
+-- Sprint 2: Core Features (6 semaines)
    +-- Guides interactifs (3 semaines) [Issue #4]
    +-- Gestion escouades (2 semaines) [Issue #5]
    +-- Calendrier raids (1 semaine) [Issue #6]

Phase 2: Améliorations (Juin - Déc 2026)
+-- Sprint 3: Gamification [Issue #7]
+-- Sprint 4: Mobile & Analytics [Issue #8]
\end{verbatim}

\textbf{Configuration GitHub Projects :}
\begin{itemize}
    \item \textbf{Colonnes :} Backlog, Sprint Planning, In Progress, Review, Done
    \item \textbf{WIP Limits :} 2 tâches max en cours par développeur
    \item \textbf{Policies :} PR obligatoire pour merge en develop
    \item \textbf{Automation :} Mise à jour automatique des statuts via GitHub Actions
\end{itemize}
\end{exemple}

\subsection{Liaison User Stories - Tests - Milestones}

\textbf{Intégration complète dans GitHub :}
\begin{itemize}
    \item \textbf{User Stories :} Créées comme issues avec template dédié
    \item \textbf{Tests :} Issues liées pour les scénarios de test
    \item \textbf{Milestones :} Regroupement logique par version
    \item \textbf{Labels :} Complexité (S, M, L, XL), type (bug, feature, docs)
\end{itemize}

\textbf{Workflow de validation :}
\begin{enumerate}
    \item Issue créée avec critères d'acceptation
    \item Branche feature développée avec tests associés
    \item Pull Request avec validation des tests automatisés
    \item Revue de code et validation manuelle si nécessaire
    \item Merge et déploiement automatique en environnement de test
\end{enumerate}

\begin{focusgithub}
\textbf{Git Flow et conventions :}
\begin{itemize}
    \item \textbf{Branches :} main, develop, feature/*, release/*, hotfix/*
    \item \textbf{PR Template :} Description, tests, checklist validation
    \item \textbf{CODEOWNERS :} Validation automatique des règles
    \item \textbf{Protection Rules :} Pas de push direct sur main/develop, status checks requis
\end{itemize}

\textbf{GitHub Projects Kanban :}
\begin{itemize}
    \item \textbf{Colonnes :} Backlog, Sprint Planning, In Progress, Review, Done
    \item \textbf{WIP Limits :} 2 features max en développement simultané
    \item \textbf{Automation :} Mise à jour statut via labels et milestones
    \item \textbf{Metrics :} Cycle time, lead time, throughput, burndown chart
\end{itemize}

\textbf{Roadmap et milestones :}
\begin{itemize}
    \item \textbf{Milestones :} v1.0 (Mai 2026), v1.1 (Juillet 2026), v1.2 (Sept 2026), v2.0 (Janvier 2027)
    \item \textbf{Dependencies :} Backend → Frontend → Tests → Déploiement
    \item \textbf{Risks :} Complexité API Bungie, performance en charge
    \item \textbf{Success Metrics :} Velocity stable, qualité code, satisfaction utilisateur
\end{itemize}
\end{focusgithub}

\begin{conseil}
\begin{itemize}
    \item Créer une roadmap GitHub réaliste avec marges
    \item Configurer GitHub Projects selon vos besoins
    \item Définir des milestones et jalons clairs
    \item Prévoir des buffers pour les imprévus
    \item Communiquer régulièrement sur l'avancement
\end{itemize}
\end{conseil}

\begin{jury}
\begin{itemize}
    \item Votre planification est-elle réaliste ?
    \item Comment gérez-vous les retards ?
    \item Quels outils utilisez-vous pour le suivi ?
    \item Comment communiquez-vous l'avancement ?
    \item Avez-vous identifié les risques de planning ?
\end{itemize}
\end{jury}

\section{Estimation de temps et planification}

Dans cette section, vous devez présenter votre estimation de temps pour chaque fonctionnalité et expliquer comment vous planifiez votre projet. Le jury attend une approche réaliste et méthodique de la gestion du temps.

\textbf{Votre estimation globale :} \textit{Le projet est estimé à 65 jours de travail effectif répartis sur 8 mois (octobre 2025 à mai 2026), incluant 20\% de marge pour les imprévus. Cette estimation couvre le développement, les tests, et le déploiement.}

L'analyse des temps permet de valider la faisabilité du projet et d'optimiser la planification selon les contraintes disponibles. Cette approche pragmatique démontre votre capacité à prendre en compte les contraintes temporelles dans les décisions techniques.

\begin{exemple}
\textbf{Estimation de temps par fonctionnalité :}
\begin{longtable}{p{2cm}p{3.2cm}p{0.9cm}p{0.6cm}p{1.6cm}p{1.7cm}}
\toprule
\textbf{Fonctionnalité} & \textbf{Description} & \textbf{Unité} & \textbf{Qty} & \textbf{Temps estimé} & \textbf{Total} \\
\midrule
Authentification & Login/Register & jours & 3 & 2 & 6 \\
Gestion projets & CRUD projets & jours & 5 & 3 & 15 \\
Tableaux de bord & Dashboard & jours & 2 & 4 & 8 \\
Tests & Tests unitaires & jours & 10 & 1 & 10 \\
Déploiement & CI/CD & jours & 3 & 2 & 6 \\
\midrule
\multicolumn{5}{r}{\textbf{Total estimé}} & \textbf{45 jours} \\
\bottomrule
\end{longtable}

\textbf{Votre estimation détaillée :} \textit{
\begin{longtable}{p{3cm}p{2.5cm}p{1.5cm}p{1.5cm}p{1.5cm}}
\toprule
\textbf{Fonctionnalité} & \textbf{Phase} & \textbf{Story Points} & \textbf{Jours estimés} & \textbf{Milestone} \\
\midrule
Environnement de développement & Setup & 3 & 3 & \href{https://github.com/xxx/milestone/1}{MVP v1.0} \\
Authentification Bungie OAuth & Backend & 5 & 5 & \href{https://github.com/xxx/milestone/1}{MVP v1.0} \\
Base de données PostgreSQL & Backend & 3 & 3 & \href{https://github.com/xxx/milestone/1}{MVP v1.0} \\
API Gestion escouades & Backend & 5 & 5 & \href{https://github.com/xxx/milestone/1}{MVP v1.0} \\
Guides interactifs raids & Frontend & 8 & 8 & \href{https://github.com/xxx/milestone/1}{MVP v1.0} \\
Calendrier raids & Frontend & 5 & 5 & \href{https://github.com/xxx/milestone/2}{v1.1} \\
Profils joueurs & Frontend & 3 & 3 & \href{https://github.com/xxx/milestone/2}{v1.1} \\
Intégration API Destiny 2 & Intégration & 8 & 8 & \href{https://github.com/xxx/milestone/2}{v1.1} \\
Recommandations équipements & Fonctionnalité & 5 & 5 & \href{https://github.com/xxx/milestone/3}{v1.2} \\
Tests unitaires et intégration & Qualité & 8 & 8 & \href{https://github.com/xxx/milestone/3}{v1.2} \\
Tests E2E & Qualité & 5 & 5 & \href{https://github.com/xxx/milestone/3}{v1.2} \\
Dockerisation & Déploiement & 3 & 3 & \href{https://github.com/xxx/milestone/4}{v1.3} \\
CI/CD & Déploiement & 5 & 5 & \href{https://github.com/xxx/milestone/4}{v1.3} \\
Documentation technique & Livraison & 3 & 3 & \href{https://github.com/xxx/milestone/4}{v1.3} \\
\midrule
\textbf{Total} & & \textbf{70} & \textbf{70 jours} & \\
\textbf{Avec marge 20\%} & & \textbf{84} & \textbf{84 jours} & \\
\bottomrule
\end{longtable}}
\end{exemple}

\subsection{Métriques de suivi et amélioration continue}

\textbf{Métriques collectées :}
\begin{itemize}
    \item \textbf{Vélocité :} Nombre de story points complétés par sprint
    \item \textbf{Burndown chart :} Progression vers l'objectif du sprint
    \item \textbf{Lead time :} Temps entre création et fermeture d'une issue
    \item \textbf{Cycle time :} Temps de développement effectif
    \item \textbf{Taux de bugs :} Nombre de bugs rapportés vs fonctionnalités livrées
\end{itemize}

\textbf{Amélioration continue :}
\begin{itemize}
    \item \textbf{Rétrospectives :} Identification points d'amélioration processus
    \item \textbf{Ajustements :} Adaptation planning basée sur la vélocité réelle
    \item \textbf{Qualité code :} Suivi couverture tests, dette technique
    \item \textbf{Satisfaction :} Retours utilisateurs sur fonctionnalités livrées
\end{itemize}

\begin{conseil}
\textbf{Ce que le jury attend dans cette section :}
\begin{itemize}
    \item Une estimation de temps réaliste et justifiée
    \item Une répartition claire par fonctionnalité
    \item Une prise en compte des phases de test et déploiement
    \item Une marge de sécurité pour les imprévus
    \item Un lien avec les user stories et milestones GitHub
\end{itemize}

\textbf{Conseils de rédaction :}
\begin{itemize}
    \item Basez-vous sur votre expérience et la complexité technique
    \item Ajoutez 20\% de marge pour les imprévus
    \item Décomposez les grosses fonctionnalités en sous-tâches
    \item Justifiez vos estimations par des arguments techniques
    \item Montrez la cohérence avec votre roadmap GitHub
\end{itemize}
\end{conseil}

\begin{jury}
\textbf{Questions de contrôle du jury :}
\begin{itemize}
    \item Comment avez-vous estimé le temps pour chaque fonctionnalité ?
    \item Avez-vous pris en compte les phases de test et déploiement ?
    \item Comment gérez-vous les dépassements de temps ?
    \item Quelle marge de sécurité avez-vous prévue ?
    \item Comment liez-vous cette estimation à vos milestones GitHub ?
    \item Que faites-vous si une fonctionnalité prend plus de temps que prévu ?
\end{itemize}
\end{jury}

\section{Liens utiles}

\begin{itemize}
    \item GitHub Project: \url{https://github.com/xxx/projects/1}
    \item CONTRIBUTING.md: \url{https://github.com/xxx/CONTRIBUTING.md}
    \item GitHub Flow/PRs: \url{https://docs.github.com/pull-requests}
    \item Git Flow: \url{https://bit.ly/gitflow-atlassian}
    \item GitHub Projects: \url{https://bit.ly/github-projects}
    \item GitHub Roadmap: \url{https://bit.ly/github-roadmap}
    \item GitHub Milestones: \url{https://bit.ly/github-milestones}
    \item User Stories: \url{https://www.mountaingoatsoftware.com/agile/user-stories}
    \item Estimation de temps: \url{https://bit.ly/time-estimation}
    \item Conventional Commits: \url{https://www.conventionalcommits.org}
\end{itemize}