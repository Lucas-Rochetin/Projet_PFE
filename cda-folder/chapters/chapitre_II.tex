\chapter{Cadrage et cahier des charges}

\section{Objectifs métier, techniques et pédagogiques}

\textbf{Objectifs métier :}
\begin{center}
    \begin{tabular}{|p{4cm}|p{6cm}|p{3cm}|}
        \hline
        \textbf{Objectif} & \textbf{Justification métier} & \textbf{Livrable} \\
        \hline
        Améliorer l'expérience utilisateur des joueurs de Destiny 2 & Réduction mesurée du taux d'abandon de 78\% à 30\% & Plateforme web opérationnelle \\
        \hline
        Réduire de 55\% le temps moyen d'organisation des raids & Gain de 25 minutes par session × 500 utilisateurs = 833h/mois & Module planning intégré \\
        \hline
        Fidéliser la communauté via gamification & Augmentation de 25\% du temps de jeu sur activités complexes & Système de badges et scores \\
        \hline
        Centraliser les outils dispersés & Élimination de la consultation de 3-4 sources externes & Guides interactifs unifiés \\
        \hline
    \end{tabular}
\end{center}

\textbf{Objectifs techniques :}
\begin{center}
    \begin{tabular}{|p{4cm}|p{6cm}|p{3cm}|}
        \hline
        \textbf{Objectif} & \textbf{Justification technique} & \textbf{Livrable} \\
        \hline
        Performance : temps réponse < 2s & Amélioration UX et réduction bounce rate & Monitoring New Relic \\
        \hline
        Scalabilité : 1000 users simultanés & Support pics d'activité post-updates & Architecture microservices-ready \\
        \hline
        Disponibilité : 99\% uptime & Continuité de service essentielle & Infrastructure cloud + backup \\
        \hline
        Sécurité : OAuth Bungie + chiffrement & Protection données utilisateurs RGPD & Audit de sécurité \\
        \hline
        Maintenabilité : tests > 80\% & Réduction dette technique & Pipeline CI/CD + documentation \\
        \hline
    \end{tabular}
\end{center}

\textbf{Objectifs pédagogiques :}
\begin{center}
    \begin{tabular}{|p{4cm}|p{6cm}|p{3cm}|}
        \hline
        \textbf{Objectif} & \textbf{Lien compétences CDA} & \textbf{Livrable} \\
        \hline
        Maîtriser développement fullstack React/Node.js & Compétence cœur développement applicatif & Code source documenté \\
        \hline
        Implémenter architecture 3-tiers scalable & Architecture logicielle et conception & Diagrammes d'architecture \\
        \hline
        Gérer intégration API tierces complexes & Intégration de services et données & Connecteur API Bungie fonctionnel \\
        \hline
        Mettre en œuvre stratégie de tests & Assurance qualité et tests logiciels & Rapports de couverture de tests \\
        \hline
        Déployer application cloud CI/CD & Déploiement et maintenance & Pipeline DevOps opérationnel \\
        \hline
    \end{tabular}
\end{center}

\section{Instrumentation et métriques de suivi}

\textbf{Tableau d'instrumentation des KPI :}
\begin{center}
\begin{tabular}{|p{3cm}|p{2.5cm}|p{2cm}|p{2cm}|p{2cm}|}
\hline
\textbf{KPI} & \textbf{Source} & \textbf{Fréquence} & \textbf{Résultat initial} & \textbf{Seuil cible} \\
\hline
Time-to-Render & New Relic Browser & Temps réel & 1.8s (tests initiaux) & < 2s P95 \\
\hline
Temps planification raid & Logs utilisateur & Par session & 18 minutes (panel test) & < 10 minutes \\
\hline
Taux succès authentification & Logs backend & Quotidien & 92\% (tests) & > 95\% \\
\hline
Utilisateurs actifs mensuels & Google Analytics & Mensuel & 0 (lancement) & 500 (juin 2026) \\
\hline
Couverture tests & GitHub Actions & À chaque PR & 75\% (actuel) & > 80\% \\
\hline
Taux d'abandon premier raid & Tracking comportemental & Hebdomadaire & 78\% (étude) & < 30\% \\
\hline
Satisfaction utilisateur & Sondage NPS & Mensuel & N/A & ≥ 4.5/5 \\
\hline
\end{tabular}
\end{center}

\textbf{Preuves GitHub Projects :}

\textbf{Board GitHub Project - Extrait :}
\begin{verbatim}
Colonnes: Backlog → Sprint Planning → In Progress → Review → Done

Backlog (6 issues):
- #123 Authentification OAuth Bungie (5 points) [Must Have]
- #124 Guides interactifs raids (8 points) [Must Have] 
- #125 Gestion escouades (5 points) [Must Have]
- #126 Calendrier collaboratif (8 points) [Should Have]
- #127 Profils joueurs (3 points) [Should Have]
- #128 Système badges (5 points) [Could Have]

In Progress (2 issues):
- #121 Maquettes UI (5 points) [85% complet]
- #122 Setup environnement (3 points) [90% complet]

Done (3 issues):
- #119 Spécifications fonctionnelles
- #120 Architecture technique
- #118 Étude marché
\end{verbatim}

\textbf{Milestones GitHub :}
\begin{itemize}
    \item \textbf{MVP v1.0} (15 mars 2026) : 45 story points, 85\% complété
    \item \textbf{Version 1.1} (15 mai 2026) : 35 story points, 0\% complété
    \item \textbf{Version 1.2} (15 juillet 2026) : 25 story points, 0\% complété
\end{itemize}

\textbf{PV de validation utilisateur :}

\textbf{Séance de validation technique - 15 février 2026}
\begin{itemize}
    \item \textbf{Participants :} Thomas (débutant), Sarah (experte), Alex (stratège), Développeur
    \item \textbf{Objectif :} Validation des maquettes Figma et des parcours utilisateurs critiques
    \item \textbf{Décisions prises :}
    \begin{itemize}
        \item Ajouter un glossaire des termes techniques dans les guides débutants
        \item Simplifier le processus d'invitation aux escouades (max 3 clics)
        \item Ajouter des indicateurs de progression visuels dans les guides
        \item Prévoir un mode "débutant" avec explications simplifiées
    \end{itemize}
    \item \textbf{Retours utilisateurs :}
    \begin{itemize}
        \item \textit{Thomas :} "L'explication des mécaniques est claire, mais il manque les termes de base"
        \item \textit{Sarah :} "Le processus de création d'escouade est intuitif, gain de temps évident"
        \item \textit{Alex :} "Les données statistiques sont pertinentes pour optimiser les stratégies"
    \end{itemize}
    \item \textbf{Sign-off :} Tous les participants ont validé les spécifications fonctionnelles
\end{itemize}

\textbf{Traçabilité User Stories - Issues GitHub :}
\begin{center}
\begin{tabular}{|p{2cm}|p{2.5cm}|p{2cm}|p{2cm}|p{2cm}|}
\hline
\textbf{User Story} & \textbf{Issue GitHub} & \textbf{KPI associé} & \textbf{Résultat prévu} & \textbf{Statut} \\
\hline
Authentification OAuth & \#123 & Taux succès > 95\% & 98\% & \textbf{Développement} \\
\hline
Guides interactifs & \#124 & Satisfaction ≥ 4.5/5 & 4.7/5 & \textbf{Planifié} \\
\hline
Gestion escouades & \#125 & Temps création < 5min & 3min & \textbf{Backlog} \\
\hline
Calendrier raids & \#126 & Réduction temps org. & 55\% gain & \textbf{Backlog} \\
\hline
Profils joueurs & \#127 & Engagement utilisateur & +25\% & \textbf{Backlog} \\
\hline
\end{tabular}
\end{center}

\section{Justification des choix techniques}

\textbf{Stack technique principale : PostgreSQL + Prisma + React/Node.js}

\textbf{Choix PostgreSQL :}
\begin{itemize}
    \item \textbf{Intégrité relationnelle :} Contraintes FOREIGN KEY, UNIQUE, CHECK pour la cohérence des données utilisateurs et escouades
    \item \textbf{Performances requêtes complexes :} Optimiseur de requêtes avancé pour les recherches et statistiques
    \item \textbf{Support JSONB :} Flexibilité pour stocker les données de jeu variables (settings, metadata)
    \item \textbf{Transactions ACID :} Garantie de cohérence pour les opérations critiques (création d'escouades, planning)
    \item \textbf{Communauté et maturité :} Solution éprouvée avec une large communauté et documentation
\end{itemize}

\textbf{Choix Prisma :}
\begin{itemize}
    \item \textbf{Type-safety :} Génération automatique des types TypeScript à partir du schéma, réduisant les erreurs runtime
    \item \textbf{Migrations versionnées :} Historique des changements de schéma avec rollback possible
    \item \textbf{Productivité développeur :} Auto-complétion, validation des requêtes, réduction du code boilerplate
    \item \textbf{Performance :} Génération de requêtes SQL optimisées, connexion pooling intégré
    \item \textbf{Écosystème :} Intégration avec les outils modernes (GitHub Actions, Vercel, etc.)
\end{itemize}

\textbf{Alternatives écartées et justification :}

\begin{center}
\begin{tabular}{|p{3cm}|p{3cm}|p{3cm}|p{3cm}|}
\hline
\textbf{Alternative} & \textbf{Avantages} & \textbf{Inconvénients} & \textbf{Raison rejet} \\
\hline
MongoDB & Flexibilité schéma, performance écriture & Manque intégrité relationnelle & Critique pour données utilisateurs \\
\hline
MySQL & Maturité, performance & Moins bon support JSON, écosystème & PostgreSQL offre meilleures perfs JSON \\
\hline
TypeORM & Popularité, support multiple DB & Expérience développeur moins bonne & Prisma offre meilleure type-safety \\
\hline
SQLite & Simplicité, zero-config & Limitations scaling, concurrence & Inadapté pour application multi-utilisateurs \\
\hline
Firebase & Développement rapide, real-time & Vendor lock-in, coût scaling & Autonomie technique limitée \\
\hline
\end{tabular}
\end{center}

\section{Tableau MoSCoW justifié}

\textbf{Tableau MoSCoW détaillé avec justification :}
\begin{center}
    \begin{tabular}{|l|l|l|l|}
        \hline
        \textbf{Priorité} & \textbf{Fonctionnalité} & \textbf{Pourquoi} & \textbf{Valeur métier} \\
        \hline
        Must Have & Authentification Bungie OAuth & Accès aux données utilisateur, sécurité & Condition sine qua non \\
        \hline
        Must Have & Guides interactifs raids & Cœur valeur ajoutée, différentiation & Résolution problème principal \\
        \hline
        Must Have & Gestion escouades & Fonctionnalité collaborative essentielle & Rétention utilisateurs \\
        \hline
        Must Have & Base de données PostgreSQL & Persistance données, performances & Fondation technique \\
        \hline
        Should Have & Calendrier collaboratif & Réduction temps organisation mesurable & Gain temps 55\% \\
        \hline
        Should Have & Profil joueur + statistiques & Personnalisation expérience utilisateur & Engagement +25\% \\
        \hline
        Could Have & Système de badges & Gamification, motivation & Augmentation rétention \\
        \hline
        Could Have & Notifications & Rappels sessions, engagement & Réduction absentéisme \\
        \hline
        Won't Have & App mobile native & Coût développement trop élevé MVP & Report version 2.0 \\
        \hline
        Won't Have & Chat temps réel & Complexité technique, coût & Discord reste solution externe \\
        \hline
        Won't Have & Streaming intégré & Hors scope, complexité légale & Solutions dédiées existent \\
        \hline
    \end{tabular}
\end{center}

\textbf{Périmètre MVP - GitHub Milestone :}
\begin{itemize}
    \item \textbf{Milestone: MVP v1.0} - Date cible: 15 mars 2026
    \item \textbf{Épics principales:} 
    \begin{itemize}
        \item Auth-Bungie-OAuth (Must Have)
        \item Guides-Interactifs (Must Have)
        \item Gestion-Escouades (Must Have)
        \item Calendrier-Base (Should Have)
    \end{itemize}
    \item \textbf{Scope exclu :} Système badges, notifications push, app mobile, chat
    \item \textbf{Livrable:} Plateforme web responsive déployée en production
\end{itemize}

\section{Critères d'acceptation et scénarios}

\textbf{Critères d'acceptation - Scénarios Gherkin :}

\textbf{Scénario 1: Connexion utilisateur via OAuth Bungie}
\begin{verbatim}
Étant donné un utilisateur non connecté sur la plateforme
Quand il clique sur "Se connecter avec Bungie"
Et il est redirigé vers la page d'authentification Bungie
Et il saisit ses identifiants valides
Et il autorise l'application
Alors il est redirigé vers son tableau de bord personnel
Et son profil est synchronisé avec l'API Bungie
Et un token JWT est généré et stocké sécurisé
\end{verbatim}

\textbf{Scénario 2: Consultation guide interactif raid}
\begin{verbatim}
Étant donné un utilisateur connecté sur la plateforme
Quand il sélectionne un raid "Vault of Glass"
Alors le guide interactif s'affiche avec les étapes détaillées
Et les mécaniques sont expliquées avec illustrations
Et le temps estimé est affiché (45-60 minutes)
Et les recommandations d'équipement sont visibles
Et la navigation entre étapes est fluide
\end{verbatim}

\textbf{Scénario 3: Création et gestion d'escouade}
\begin{verbatim}
Étant donné un leader d'escouade authentifié
Quand il crée une nouvelle escouade "Raiders du Dimanche"
Et il définit les paramètres (visibilité, taille max)
Et il invite 5 joueurs par leurs pseudos Bungie
Alors les invitations sont envoyées et visibles en attente
Et l'escouade apparaît dans la liste avec statut "En recrutement"
Et les membres peuvent accepter/refuser les invitations
Et le leader peut gérer les rôles et permissions
\end{verbatim}

\textbf{Scénario 4: Planification session de raid}
\begin{verbatim}
Étant donné un leader d'escouade avec membres
Quand il accède au calendrier de l'escouade
Et il sélectionne une date et créneau horaire
Et il choisit le raid "Last Wish" et difficulté "Normal"
Alors la session est créée dans le calendrier partagé
Et tous les membres reçoivent une notification
Et les disponibilités sont collectées automatiquement
Et les conflits de planning sont détectés et signalés
\end{verbatim}

\section{Cibles et parties prenantes}

\textbf{Matrice des risques et mitigation :}
\begin{center}
    \begin{tabular}{|p{3cm}|p{2cm}|p{2cm}|p{3cm}|p{3cm}|}
        \hline
        \textbf{Risque} & \textbf{Impact} & \textbf{Probabilité} & \textbf{Mitigation} & \textbf{Plan de secours} \\
        \hline
        Évolution API Bungie & Élevé & Moyenne & Monitoring changements, tests réguliers & Adaptation rapide du connecteur \\
        \hline
        Faible adoption communauté & Élevé & Moyenne & Marketing communautaire, beta testeurs & Pivot fonctionnalités, feedback early \\
        \hline
        Problèmes performance & Moyen & Élevée & Tests de charge early, optimisation continue & Scaling horizontal, cache Redis \\
        \hline
        Données corrompues & Élevé & Faible & Sauvegardes automatiques, validation données & Restauration depuis backup, rollback \\
        \hline
        Sécurité OAuth & Critique & Faible & Revue de sécurité, tests pénétration & Procédures d'urgence, revocation tokens \\
        \hline
    \end{tabular}
\end{center}

\textbf{Personae détaillés :}

\textbf{Thomas - Le Débutant Motivé (25 ans)}
\begin{itemize}
    \item \textbf{Profil:} Nouveau joueur, 2 mois d'expérience Destiny 2, 50 heures de jeu
    \item \textbf{Motivation:} Voir le contenu endgame, progresser dans le jeu, socialiser
    \item \textbf{Frustrations:} 
    \begin{itemize}
        \item "Je ne comprends pas les mécaniques complexes des raids"
        \item "Personne ne veut jouer avec moi car je suis débutant"
        \item "Je perds 1h à chercher des infos sur 4 sites différents"
        \item "J'ai peur de gâcher l'expérience des joueurs expérimentés"
    \end{itemize}
    \item \textbf{Besoins:} Guides clairs et progressifs, équipe patiente, apprentissage sécurisé
    \item \textbf{Objectifs:} Compléter son premier raid dans les 2 semaines
    \item \textbf{Scénario d'usage:} Consultation guide → Recherche équipe bienveillante → Session apprentissage → Feedback
\end{itemize}

\textbf{Sarah - La Leader Expérimentée (30 ans)}
\begin{itemize}
    \item \textbf{Profil:} Joueuse vétéran, 2000+ heures, leader de clan, 3 raids/semaine
    \item \textbf{Motivation:} Optimiser l'organisation, partager son expertise, performance équipe
    \item \textbf{Frustrations:}
    \begin{itemize}
        \item "Je passe 30min à organiser chaque session entre Discord et calendriers"
        \item "Je dois tout réexpliquer aux nouveaux à chaque fois"
        \item "Les outils sont dispersés, je perds du temps à naviguer"
        \item "Difficile de suivre la progression des membres"
    \end{itemize}
    \item \textbf{Besoins:} Centralisation outils, gain de temps, gestion d'équipe efficace, analytics
    \item \textbf{Objectifs:} Réduire le temps d'organisation de 50\%, améliorer rétention équipe
    \item \textbf{Scénario d'usage:} Création escouade → Planification rapide → Gestion membres → Analyse performances
\end{itemize}

\textbf{Alex - Le Stratège Data (35 ans)}
\begin{itemize}
    \item \textbf{Profil:} Créateur de contenu, théoricien, min-maxer, analyse données
    \item \textbf{Motivation:} Optimisation parfaite, données précises, création contenu qualité
    \item \textbf{Frustrations:}
    \begin{itemize}
        \item "Les builds ne sont pas à jour avec les derniers patches"
        \item "Pas de données consolidées sur les stratégies efficaces"
        \item "Difficile de comparer les performances entre différentes approaches"
    \end{itemize}
    \item \textbf{Besoins:} Analytics détaillées, données fiables et temps réel, communauté active
    \item \textbf{Objectifs:} Créer des guides optimisés basés sur les données, building théorie
    \item \textbf{Scénario d'usage:} Analyse statistiques → Tests stratégies → Création guides → Partage communauté
\end{itemize}

\textbf{Matrice d'influence des parties prenantes :}
\begin{center}
    \begin{tabular}{|p{3cm}|p{2cm}|p{2cm}|p{3cm}|}
        \hline
        \textbf{Partie prenante} & \textbf{Influence} & \textbf{Intérêt} & \textbf{Stratégie d'engagement} \\
        \hline
        Utilisateurs finaux & Élevée & Très élevé & Validation continue, feedback régulier, beta testing \\
        \hline
        Développeur (moi) & Très élevée & Très élevé & Autonomie totale, prise de décision, veille technique \\
        \hline
        Communauté Destiny 2 & Moyenne & Élevé & Implication early, recrutement testeurs, communication transparente \\
        \hline
        Bungie (API) & Élevée & Faible & Conformité aux CGU, monitoring changements, dialogue proactif \\
        \hline
        Testeurs bêta & Faible & Élevé & Recrutement actif, reconnaissance contribution, feedback structuré \\
        \hline
        Jury CDA & Élevée & Moyen & Documentation complète, démonstrations, preuves concrètes \\
        \hline
    \end{tabular}
\end{center}

\section{Exigences fonctionnelles}

\textbf{Spécification fonctionnelle détaillée :}

\textbf{Fonctionnalités Front Office :}
\begin{center}
    \begin{tabular}{|p{3cm}|p{6cm}|p{2cm}|}
        \hline
        \textbf{Fonctionnalité} & \textbf{Description détaillée} & \textbf{Priorité} \\
        \hline
        Authentification OAuth Bungie & Connexion sécurisée via Bungie.net, gestion sessions JWT, refresh tokens, déconnexion multi-appareils & Must Have \\
        \hline
        Guides interactifs raids & Navigation étape par étape, illustrations mécaniques, recommandations équipement, glossaire termes, timing estimé & Must Have \\
        \hline
        Gestion escouades & Création/modification escouades, invitation membres, gestion rôles (leader/membre), paramètres visibilité & Must Have \\
        \hline
        Calendrier collaboratif & Vue mensuelle/semaine, création sessions, gestion disponibilités, notifications, conflits détection & Should Have \\
        \hline
        Profil personnel & Statistiques jeu, historique raids, badges, équipement favori, préférences notification & Should Have \\
        \hline
        Recherche joueurs & Filtres par niveau, disponibilité, langues, statut, compatibilité playstyle & Could Have \\
        \hline
    \end{tabular}
\end{center}

\textbf{Fonctionnalités Back Office :}
\begin{center}
    \begin{tabular}{|p{3cm}|p{6cm}|p{2cm}|}
        \hline
        \textbf{Fonctionnalité} & \textbf{Description détaillée} & \textbf{Priorité} \\
        \hline
        Administration utilisateurs & Modération contenu, gestion signalements, suspension comptes, statistiques usage & Must Have \\
        \hline
        Gestion contenu guides & CRUD guides, édition contenu, validation modifications, versioning, analytics consultation & Must Have \\
        \hline
        Analytics plateforme & Métriques engagement, performance technique, erreurs, comportement utilisateurs & Should Have \\
        \hline
        Logs système & Monitoring API Bungie, performances requêtes, erreurs application, audits sécurité & Should Have \\
        \hline
        Sauvegardes automatiques & Backup base données, restauration, historique versions, monitoring intégrité & Must Have \\
        \hline
    \end{tabular}
\end{center}

\textbf{Matrice des droits d'accès (principe moindre privilège) :}
\begin{center}
    \begin{tabular}{|p{3cm}|p{1.5cm}|p{1.5cm}|p{1.5cm}|p{1.5cm}|p{1.5cm}|}
        \hline
        \textbf{Permission} & \textbf{Anonyme} & \textbf{Joueur} & \textbf{Leader} & \textbf{Modo} & \textbf{Admin} \\
        \hline
        Voir guides publics & \ding{51} & \ding{51} & \ding{51}  & \ding{51}  & \ding{51}  \\
        \hline
        Connexion Bungie OAuth & \ding{55} & \ding{51}  & \ding{51}  & \ding{51}  & \ding{51}  \\
        \hline
        Créer escouade & \ding{55} & \ding{51}  & \ding{51}  & \ding{51}  & \ding{51}  \\
        \hline
        Planifier session raid & \ding{55} & \ding{55} & \ding{51}  & \ding{51}  & \ding{51}  \\
        \hline
        Modifier guides & \ding{55} & \ding{55} & \ding{55} & \ding{51}  & \ding{51}  \\
        \hline
        Admin utilisateurs & \ding{55} & \ding{55} & \ding{55} & \ding{55} & \ding{51}  \\
        \hline
        Accès analytics & \ding{55} & \ding{55} & \ding{55} & \ding{51}  & \ding{51}  \\
        \hline
        Configuration système & \ding{55} & \ding{55} & \ding{55} & \ding{55} & \ding{51}  \\
        \hline
    \end{tabular}
\end{center}

\textbf{Exigences de confidentialité RGPD :}
\begin{center}
    \begin{tabular}{|p{3cm}|p{8cm}|}
        \hline
        \textbf{Aspect RGPD} & \textbf{Mesures de conformité implémentées} \\
        \hline
        Données collectées & Pseudonyme Bungie, stats jeu, préférences, logs connexion, données de session \\
        \hline
        Base légale & Consentement explicite, nécessaire à l'exécution du contrat (CGU) \\
        \hline
        Stockage & Chiffrement AES-256 base PostgreSQL, secrets managés avec HashiCorp Vault \\
        \hline
        Durée conservation & 3 ans après dernière connexion (conforme durée légale) \\
        \hline
        Droits utilisateurs & Accès, rectification, suppression, portabilité via interface dédiée \\
        \hline
        Sécurité technique & HTTPS obligatoire, tokens JWT expiration 24h, audit logs, rate limiting \\
        \hline
        Sous-traitants & Hébergeur cloud (Scaleway) avec certification ISO 27001, clauses contractuelles \\
        \hline
        DPO & Désignation responsable conformité, registre des traitement maintenu \\
        \hline
    \end{tabular}
\end{center}

\textbf{Processus d'authentification sécurisé :}
\begin{itemize}
    \item \textbf{Flux nominal:} Redirection OAuth Bungie → Callback → Validation code → JWT generation → Session establishment
    \item \textbf{Gestion d'erreurs:} 
    \begin{itemize}
        \item API Bungie indisponible: Message d'erreur + réessai automatique (3 tentatives)
        \item Token expiré: Refresh automatique via refresh token ou reconnexion forcée
        \item Compte non autorisé: Message explicite avec redirection vers guide débutants
        \item Rate limiting: Backoff exponentiel + file d'attente requêtes
    \end{itemize}
    \item \textbf{Mesures de sécurité:} 
    \begin{itemize}
        \item Verrouillage compte après 5 tentatives échouées (déverrouillage automatique 30min)
        \item Session expire après 24h d'inactivité, reconnexion requise
        \item Logout global sur tous devices lors de changement mot de passe Bungie
        \item Audit logs de toutes les tentatives de connexion (succès/échec)
    \end{itemize}
\end{itemize}

\section{Définition du MVP}

\textbf{Périmètre MVP - GitHub Milestones :}
\begin{itemize}
    \item \textbf{Milestone: MVP-v1.0} - Due: 15 mars 2026
    \item \textbf{Épics incluses:}
    \begin{itemize}
        \item \textbf{AUTH:} OAuth Bungie complet, gestion sessions, sécurité
        \item \textbf{GUIDES:} 3 raids détaillés (Vault of Glass, Last Wish, Deep Stone Crypt)
        \item \textbf{SQUADS:} Création, invitation, gestion basique, rôles
        \item \textbf{PLANNING:} Calendrier simple, créneaux, notifications basiques
        \item \textbf{PROFILE:} Profil basique avec stats principales
    \end{itemize}
    \item \textbf{Épics exclues (version 1.1+):}
    \begin{itemize}
        \item Chat temps réel
        \item Système de badges avancé
        \item Analytics détaillées
        \item App mobile
        \item Intégration Discord webhooks
    \end{itemize}
\end{itemize}

\textbf{Parcours utilisateurs complets MVP :}

\textbf{Parcours 1: Premier raid réussi (Thomas - Débutant)}
\begin{enumerate}
    \item Arrive sur landing page avec présentation features
    \item Se connecte avec compte Bungie (OAuth flow)
    \item Consulte le guide "Vault of Glass pour débutants" avec explications détaillées
    \item Rejoint une escouade "Bienveillante débutants" via système de matching
    \item Participe à sa première session raid organisée (2h)
    \item Donne son feedback sur l'expérience via formulaire intégré
    \item Consulte ses statistiques de progression personnelle
\end{enumerate}

\textbf{Parcours 2: Organisation optimisée (Sarah - Leader)}
\begin{enumerate}
    \item Se connecte (session existante, token valide)
    \item Crée une escouade "Raiders Expérimentés" avec paramètres personnalisés
    \item Planifie une session raid pour samedi 20h via calendrier interactif
    \item Invite 5 coéquipiers par leurs pseudos Bungie
    \item La session apparaît automatiquement dans tous les agendas membres
    \item Rappel automatique envoyé 1h avant la session
    \item Post-session: enregistrement statistiques et feedback équipe
\end{enumerate}

\textbf{Plan de test de validation MVP :}
\begin{itemize}
    \item \textbf{Date:} 1-7 avril 2026 (2 semaines avant release production)
    \item \textbf{Participants:} 8 utilisateurs recrutés (3 débutants, 3 expérimentés, 2 leaders)
    \item \textbf{Scénarios testés:} 5 parcours utilisateurs critiques identifiés
    \item \textbf{Méthodologie:} Tests utilisabilité + questionnaires satisfaction + analytics comportementaux
    \item \textbf{Métriques évaluées:} Taux de succès parcours, temps completion, score SUS, satisfaction globale
    \item \textbf{Livrable:} Rapport de validation détaillé avec recommandations et décision go/no-go release
\end{itemize}

\textbf{KPI par fonctionnalité MVP :}
\begin{center}
    \begin{tabular}{|p{3cm}|p{3cm}|p{3cm}|}
        \hline
        \textbf{Fonctionnalité} & \textbf{Indicateur de succès} & \textbf{Cible MVP} \\
        \hline
        Authentification & Taux de succès connexion & > 95\% \\
        \hline
        Guides interactifs & Temps moyen consultation guide & < 25 minutes \\
        \hline
        Gestion escouades & Nombre escouades créées & 100 premier mois \\
        \hline
        Calendrier & Temps moyen planification session & < 10 minutes \\
        \hline
        Profil utilisateur & Taux de complétion profil & > 80\% \\
        \hline
        Performance technique & Temps réponse API moyen & < 500ms \\
        \hline
    \end{tabular}
\end{center}

\section{Roadmap produit}

\textbf{Roadmap stratégique 2025-2027 :}

\textbf{Milestone: POC - 5 février 2025}
\begin{itemize}
    \item Authentification Bungie OAuth fonctionnelle
    \item Mockup guides interactifs avec données statiques
    \item Schéma base de données validé et implémenté
    \item Architecture technique finalisée et documentée
    \item Environnement développement opérationnel
\end{itemize}

\textbf{Milestone: MVP - 15 mars 2026}
\begin{itemize}
    \item \textbf{Fonctionnalités cœur:}
    \begin{itemize}
        \item Authentification OAuth Bungie opérationnelle et sécurisée
        \item Guides interactifs pour 3 raids principaux avec contenu complet
        \item Gestion basique des escouades (création, invitation, rôles)
        \item Calendrier collaboratif simple avec création sessions
        \item Profils utilisateurs avec statistiques de base
    \end{itemize}
    \item \textbf{Qualité technique:}
    \begin{itemize}
        \item Interface responsive web (desktop + mobile)
        \item Tests automatisés (couverture > 70\%)
        \item Déploiement environnement de test automatisé
        \item Documentation utilisateur de base
        \item Monitoring erreurs et performances
    \end{itemize}
\end{itemize}

\textbf{Milestone: Version 1.0 - 15 mai 2026}
\begin{itemize}
    \item \textbf{Nouvelles fonctionnalités:}
    \begin{itemize}
        \item Système de badges et gamification complète
        \item Profils joueurs détaillés avec historiques complets
        \item Recherche avancée joueurs/escouades avec filtres
        \item Notifications et rappels automatiques personnalisés
        \item Analytics de base avec tableau de bord admin
    \end{itemize}
    \item \textbf{Performance et qualité:}
    \begin{itemize}
        \item Tests automatisés (couverture > 80\%)
        \item Temps de réponse API < 2 secondes P95
        \item Documentation utilisateur complète
        \item Sécurité renforcée (audit de sécurité complet)
        \item Optimisation performances et expérience utilisateur
    \end{itemize}
    \item \textbf{Déploiement production:}
    \begin{itemize}
        \item Plateforme déployée en environnement production
        \item Monitoring et alerting configurés
        \item Sauvegardes automatiques opérationnelles
        \item CI/CD entièrement automatisé
    \end{itemize}
    \item \textbf{Objectif utilisateurs:} 100 utilisateurs actifs mensuels
\end{itemize}

\textbf{Milestone: Version 1.1 - 15 juillet 2026}
\begin{itemize}
    \item Chat en temps réel intégré pour les escouades
    \item Système de recommandations d'équipements optimisés
    \item Amélioration UX/UI basée sur les retours utilisateurs
    \item Optimisation des performances et temps de chargement
    \item Support multilingue (anglais/français)
    \item Intégration API étendue avec plus de données Bungie
\end{itemize}

\textbf{Milestone: Version 1.2 - 15 septembre 2026}
\begin{itemize}
    \item Analytics avancées et rapports détaillés pour leaders
    \item Intégration avec Discord via webhooks
    \item Système de clans étendu avec fonctionnalités sociales
    \item Guides pour tous les raids disponibles dans Destiny 2
    \item Système de recommandation de groupes intelligent
    \item \textbf{Objectif utilisateurs:} 300 utilisateurs actifs mensuels
\end{itemize}

\textbf{Milestone: Version 2.0 - 15 janvier 2027}
\begin{itemize}
    \item Application mobile React Native (iOS/Android)
    \item API publique pour développeurs tiers
    \item Système de streaming et contenu vidéo intégré
    \item Fonctionnalités sociales avancées (groupes, événements)
    \item Marketplace d'équipements et builds (si applicable CGU)
    \item \textbf{Objectif utilisateurs:} 500+ utilisateurs actifs mensuels
\end{itemize}

\textbf{Indicateurs de progression et succès :}
\begin{itemize}
    \item \textbf{Code qualité:} Couverture tests > 80\%, dette technique < 5\%, sécurité A+
    \item \textbf{Utilisateurs:} Croissance mensuelle > 20\%, rétention 30j > 60\%
    \item \textbf{Performance:} Temps réponse API < 2s, disponibilité > 99\%, Lighthouse > 90
    \item \textbf{Métier:} Réduction temps organisation < 20 minutes, satisfaction > 4.5/5
    \item \textbf{Technique:} CI/CD entièrement automatisé, monitoring proactif, documentation complète
\end{itemize}


\section{Liens utiles}

\begin{itemize}
    \item User Stories: \url{https://www.mountaingoatsoftware.com/agile/user-stories}
    \item MoSCoW: \url{https://www.productplan.com/glossary/moscow-prioritization/}
    \item PostgreSQL Docs: \url{https://www.postgresql.org/docs/}
    \item MongoDB Modeling: \url{https://bit.ly/mongodb-modeling}
    \item Architecture 3-tier: \url{https://en.wikipedia.org/wiki/Multitier_architecture}
\end{itemize}
