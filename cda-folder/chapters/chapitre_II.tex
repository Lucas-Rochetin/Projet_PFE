\chapter{Cadrage et cahier des charges}

\section{Objectifs métier, techniques et pédagogiques}

\textbf{Objectifs métier :}
\begin{center}
    \begin{tabular}{|p{4cm}|p{6cm}|p{3cm}|}
        \hline
        \textbf{Objectif} & \textbf{Justification métier} & \textbf{Livrable} \\
        \hline
        Améliorer l'expérience utilisateur des joueurs de Destiny 2 & Réduction mesurée du taux d'abandon de 78\% à 30\% & Plateforme web opérationnelle \\
        \hline
        Réduire de 55\% le temps moyen d'organisation des raids & Gain de 25 minutes par session × 500 utilisateurs = 833h/mois & Module planning intégré \\
        \hline
        Fidéliser la communauté via gamification & Augmentation de 25\% du temps de jeu sur activités complexes & Système de badges et scores \\
        \hline
        Centraliser les outils dispersés & Élimination de la consultation de 3-4 sources externes & Guides interactifs unifiés \\
        \hline
    \end{tabular}
\end{center}

\textbf{Objectifs techniques :}
\begin{center}
    \begin{tabular}{|p{4cm}|p{6cm}|p{3cm}|}
        \hline
        \textbf{Objectif} & \textbf{Justification technique} & \textbf{Livrable} \\
        \hline
        Performance : temps réponse < 2s & Amélioration UX et réduction bounce rate & Monitoring New Relic \\
        \hline
        Scalabilité : 1000 users simultanés & Support pics d'activité post-updates & Architecture microservices-ready \\
        \hline
        Disponibilité : 99\% uptime & Continuité de service essentielle & Infrastructure cloud + backup \\
        \hline
        Sécurité : OAuth Bungie + chiffrement & Protection données utilisateurs RGPD & Audit de sécurité \\
        \hline
        Maintenabilité : tests > 80\% & Réduction dette technique & Pipeline CI/CD + documentation \\
        \hline
    \end{tabular}
\end{center}

\textbf{Objectifs pédagogiques :}
\begin{center}
    \begin{tabular}{|p{4cm}|p{6cm}|p{3cm}|}
        \hline
        \textbf{Objectif} & \textbf{Lien compétences CDA} & \textbf{Livrable} \\
        \hline
        Maîtriser développement fullstack React/Node.js & Compétence cœur développement applicatif & Code source documenté \\
        \hline
        Implémenter architecture 3-tiers scalable & Architecture logicielle et conception & Diagrammes d'architecture \\
        \hline
        Gérer intégration API tierces complexes & Intégration de services et données & Connecteur API Bungie fonctionnel \\
        \hline
        Mettre en œuvre stratégie de tests & Assurance qualité et tests logiciels & Rapports de couverture de tests \\
        \hline
        Déployer application cloud CI/CD & Déploiement et maintenance & Pipeline DevOps opérationnel \\
        \hline
    \end{tabular}
\end{center}

\textbf{Tableau MoSCoW justifié :}
\begin{center}
    \begin{tabular}{|l|l|l|l|}
        \hline
        \textbf{Priorité} & \textbf{Fonctionnalité} & \textbf{Pourquoi} & \textbf{Valeur} \\
        \hline
        Must Have & Authentification Bungie & Sécurité + accès données utilisateur & Condition sine qua non \\
        \hline
        Must Have & Guides interactifs raids & Cœur valeur ajoutée + différentiation & Résolution problème principal \\
        \hline
        Must Have & Gestion escouades & Fonctionnalité collaborative essentielle & Rétention utilisateurs \\
        \hline
        Should Have & Calendrier collaboratif & Réduction temps organisation & Gain temps mesurable \\
        \hline
        Should Have & Profil joueur + stats & Personnalisation expérience & Engagement utilisateur \\
        \hline
        Could Have & Système de badges & Gamification & Augmentation rétention \\
        \hline
        Could Have & Chat temps réel & Communication équipe & Plus-value collaborative \\
        \hline
        Won't Have & App mobile native & Coût développement trop élevé MVP & Report version 2.0 \\
        \hline
    \end{tabular}
\end{center}

\textbf{Périmètre MVP - Milestone GitHub :}
\begin{itemize}
    \item \textbf{Milestone: MVP v1.0} - Date cible: 15 mars 2026
    \item \textbf{Épics principales:} 
    \begin{itemize}
        \item Auth-Bungie-OAuth
        \item Guides-Interactifs
        \item Gestion-Escouades
        \item Calendrier-Base
    \end{itemize}
    \item \textbf{Livrable:} Plateforme web déployée en production
\end{itemize}

\textbf{Critères d'acceptation - Scénarios Gherkin :}

\textbf{Scénario 1: Connexion utilisateur}
\begin{verbatim}
    Étant donné un utilisateur non connecté
    Quand il clique sur "Se connecter avec Bungie"
    Et il valide l'autorisation OAuth
    Alors il est redirigé vers son tableau de bord
    Et son profil est synchronisé avec l'API Bungie
\end{verbatim}

\textbf{Scénario 2: Consultation guide raid}
\begin{verbatim}
    Étant donné un utilisateur connecté
    Quand il sélectionne un raid "Vault of Glass"
    Alors le guide interactif s'affiche avec étapes détaillées
    Et les mécaniques sont expliquées avec illustrations
    Et le temps estimé est affiché (45-60 minutes)
\end{verbatim}

\textbf{Scénario 3: Création escouade}
\begin{verbatim}
    Étant donné un leader d'escouade
    Quand il crée une nouvelle escouade "Raiders du Dimanche"
    Et il invite 5 joueurs par leurs pseudos
    Alors les invitations sont envoyées
    Et l'escouade apparaît dans le calendrier
\end{verbatim}

\section{Cibles et parties prenantes}

\textbf{Matrice des risques et mitigation :}
\begin{center}
    \begin{tabular}{|p{3cm}|p{2cm}|p{2cm}|p{3cm}|p{3cm}|}
        \hline
        \textbf{Risque} & \textbf{Impact} & \textbf{Probabilité} & \textbf{Mitigation} & \textbf{Plan de secours} \\
        \hline
        Évolution API Bungie & Élevé & Moyenne & Monitoring changements & Adaptation rapide du connecteur \\
        \hline
        Faible adoption & Élevé & Moyenne & Marketing communautaire & Pivot fonctionnalités \\
        \hline
        Problèmes performance & Moyen & Élevée & Tests de charge early & Optimisation progressive \\
        \hline
        Données corrompues & Élevé & Faible & Sauvegardes automatiques & Restauration depuis backup \\
        \hline
    \end{tabular}
\end{center}

\textbf{Personae détaillés :}

\textbf{Thomas - Le Débutant Motivé (25 ans)}
\begin{itemize}
    \item \textbf{Profil:} Nouveau joueur, 2 mois d'expérience Destiny 2
    \item \textbf{Motivation:} Voir le contenu endgame, progresser dans le jeu
    \item \textbf{Frustrations:} 
    \begin{itemize}
        \item "Je ne comprends pas les mécaniques complexes"
        \item "Personne ne veut jouer avec moi car je suis débutant"
        \item "Je perds 1h à chercher des infos sur 4 sites différents"
    \end{itemize}
    \item \textbf{Besoins:} Guides clairs, équipe patiente, apprentissage progressif
    \item \textbf{Objectifs:} Compléter son premier raid dans les 2 semaines
\end{itemize}

\textbf{Sarah - La Leader Expérimentée (30 ans)}
\begin{itemize}
    \item \textbf{Profil:} Joueuse vétéran, 2000+ heures, leader de clan
    \item \textbf{Motivation:} Optimiser l'organisation, partager son expertise
    \item \textbf{Frustrations:}
    \begin{itemize}
        \item "Je passe 30min à organiser chaque session"
        \item "Je dois tout réexpliquer aux nouveaux"
        \item "Les outils sont dispersés (Discord, Google Calendar...)"
    \end{itemize}
    \item \textbf{Besoins:} Centralisation, gain de temps, gestion d'équipe
    \item \textbf{Objectifs:} Réduire le temps d'organisation de 50\%
\end{itemize}

\textbf{Alex - Le Stratège (35 ans)}
\begin{itemize}
    \item \textbf{Profil:} Créateur de contenu, théoricien, min-maxer
    \item \textbf{Motivation:} Optimisation parfaite, données précises
    \item \textbf{Frustrations:}
    \begin{itemize}
        \item "Les builds ne sont pas à jour avec les patches"
        \item "Pas de données consolidées sur les stratégies"
    \end{itemize}
    \item \textbf{Besoins:} Analytics, données fiables, communauté active
    \item \textbf{Objectifs:} Créer des guides optimisés basés sur les données
\end{itemize}

\textbf{Matrice d'influence des parties prenantes :}
\begin{center}
    \begin{tabular}{|p{3cm}|p{2cm}|p{2cm}|p{3cm}|}
        \hline
        \textbf{Partie prenante} & \textbf{Influence} & \textbf{Intérêt} & \textbf{Stratégie} \\
        \hline
        Utilisateurs finaux & Élevée & Très élevé & Validation continue, feedback régulier \\
        \hline
        Développeur (moi) & Très élevée & Très élevé & Autonomie totale, prise de décision \\
        \hline
        Communauté Destiny 2 & Moyenne & Élevé & Implication early, beta testing \\
        \hline
        Bungie (API) & Élevée & Faible & Conformité aux CGU, monitoring changements \\
        \hline
        Testeurs beta & Faible & Élevé & Recrutement actif, reconnaissance \\
        \hline
    \end{tabular}
\end{center}

\textbf{User Stories formatées :}

\textbf{US1: Consultation guide débutant}
\begin{verbatim}
    En tant que Thomas (débutant)
    Je veux consulter un guide raid étape par étape
    Afin de comprendre les mécaniques sans être submergé
    
    Critères d'acceptation:
    - Given un débutant sur la page guide
    - When il sélectionne une étape
    - Then l'explication s'affiche avec illustration
    - And les termes techniques sont définis
    - And le temps estimé est visible
\end{verbatim}

\textbf{US2: Organisation session}
\begin{verbatim}
    En tant que Sarah (leader expérimentée)
    Je veux planifier une session raid en 3 clics
    Afin de gagner du temps sur l'organisation
    
    Critères d'acceptation:
    - Given une leader connectée
    - When elle crée une nouvelle session
    - Then le calendrier propose des créneaux
    - And les membres disponibles sont notifiés
    - And la session apparaît dans l'agenda
\end{verbatim}

\textbf{Plan de validation utilisateur :}
\begin{itemize}
    \item \textbf{Session test MVP:} 5 utilisateurs recrutés dans la communauté
    \item \textbf{Date:} 1er avril 2026 (2 semaines avant release)
    \item \textbf{Méthodologie:} Tests utilisabilité + questionnaires satisfaction
    \item \textbf{Métriques:} Tâches accomplies, temps completion, score SUS
    \item \textbf{Livrable:} Rapport de tests avec recommandations
\end{itemize}

\section{Exigences fonctionnelles}

\textbf{Spécification fonctionnelle détaillée :}

\textbf{Fonctionnalités Front Office :}
\begin{center}
    \begin{tabular}{|p{3cm}|p{6cm}|p{2cm}|}
        \hline
        \textbf{Fonctionnalité} & \textbf{Description} & \textbf{Priorité} \\
        \hline
        Authentification & OAuth Bungie, gestion sessions, profil sync & Must Have \\
        \hline
        Guides interactifs & Étapes détaillées, illustrations, termes définis & Must Have \\
        \hline
        Gestion escouades & Création, invitation, rôles, calendrier & Must Have \\
        \hline
        Recherche joueurs & Filtres par niveau, disponibilité, langues & Should Have \\
        \hline
        Profil personnel & Stats, badges, historique, équipement & Should Have \\
        \hline
        Calendrier & Sessions planifiées, rappels, disponibilités & Should Have \\
        \hline
    \end{tabular}
\end{center}

\textbf{Fonctionnalités Back Office :}
\begin{center}
    \begin{tabular}{|p{3cm}|p{6cm}|p{2cm}|}
        \hline
        \textbf{Fonctionnalité} & \textbf{Description} & \textbf{Priorité} \\
        \hline
        Admin utilisateurs & Modération, stats d'usage, support & Must Have \\
        \hline
        Gestion contenu & CRUD guides, illustrations, mises à jour & Must Have \\
        \hline
        Analytics & Métriques engagement, performance, erreurs & Should Have \\
        \hline
        Logs système & Monitoring API, erreurs, performances & Should Have \\
        \hline
        Backup auto & Sauvegardes base de données, restore & Must Have \\
        \hline
    \end{tabular}
\end{center}

\textbf{Matrice des droits d'accès (principe moindre privilège) :}
\begin{center}
    \begin{tabular}{|p{3cm}|p{1.5cm}|p{1.5cm}|p{1.5cm}|p{1.5cm}|p{1.5cm}|}
        \hline
        \textbf{Permission} & \textbf{Anonyme} & \textbf{Joueur} & \textbf{Leader} & \textbf{Modo} & \textbf{Admin} \\
        \hline
        Voir guides publics & \ding{51} & \ding{51} & \ding{51}  & \ding{51}  & \ding{51}  \\
        \hline
        Connexion Bungie & \ding{55} & \ding{51}  & \ding{51}  & \ding{51}  & \ding{51}  \\
        \hline
        Crée escouade & \ding{55} & \ding{51}  & \ding{51}  & \ding{51}  & \ding{51}  \\
        \hline
        Modifier guides & \ding{55} & \ding{55} & \ding{55} & \ding{51}  & \ding{51}  \\
        \hline
        Admin utilisateurs & \ding{55} & \ding{55} & \ding{55} & \ding{55} & \ding{51}  \\
        \hline
        Accès analytics & \ding{55} & \ding{55} & \ding{55} & \ding{51}  & \ding{51}  \\
        \hline
    \end{tabular}
\end{center}

\textbf{Exigences de confidentialité RGPD :}
\begin{center}
    \begin{tabular}{|p{3cm}|p{8cm}|}
        \hline
        \textbf{Aspect} & \textbf{Mesures de conformité} \\
        \hline
        Données collectées & Pseudonyme Bungie, stats jeu, préférences, logs connexion \\
        \hline
        Stockage & Chiffrement AES-256, base PostgreSQL sécurisée \\
        \hline
        Durée conservation & 3 ans après dernière connexion (durée légale) \\
        \hline
        Droits utilisateurs & Accès, rectification, suppression, portabilité via interface \\
        \hline
        Sécurité & HTTPS obligatoire, tokens JWT expiration 24h, audit logs \\
        \hline
        Sous-traitants & Hébergeur cloud (Scaleway) avec certification ISO 27001 \\
        \hline
    \end{tabular}
\end{center}

\textbf{Processus d'authentification sécurisé :}
\begin{itemize}
    \item \textbf{Flux nominal:} Redirection OAuth Bungie $\rightarrow$ Callback $\rightarrow$ JWT generation
    \item \textbf{Erreurs:} 
    \begin{itemize}
        \item API Bungie indisponible: Message d'erreur + réessai automatique
        \item Token expiré: Refresh automatique ou reconnexion
        \item Compte non autorisé: Message explicite pour nouveaux joueurs
    \end{itemize}
    \item \textbf{Sécurité:} 
    \begin{itemize}
        \item Verrouillage après 5 tentatives échouées (30min)
        \item Session expire après 24h d'inactivité
        \item Logout global sur tous devices
    \end{itemize}
\end{itemize}

\textbf{Revue d'exigences avec utilisateurs :}
\begin{itemize}
    \item \textbf{Date:} 15 février 2026
    \item \textbf{Participants:} 3 joueurs test (débutant, expérimenté, leader)
    \item \textbf{Méthode:} Atelier de conception collaborative (design studio)
    \item \textbf{Livrable:} PV de revue avec modifications validées
    \item \textbf{Validation:} Sign-off des spécifications fonctionnelles
\end{itemize}

\section{Exigences et choix techniques}

\textbf{Justification des choix techniques - DECISIONS.md :}

\textbf{Choix: React/Node.js/PostgreSQL}
\begin{itemize}
    \item \textbf{Alternatives considérées:} 
    \begin{itemize}
        \item Vue.js/Laravel/MySQL (écosystème PHP mature)
        \item Angular/.NET/SQL Server (stack enterprise)
    \end{itemize}
    \item \textbf{Raison du choix:} 
    \begin{itemize}
        \item Stack JavaScript unifiée (réutilisation compétences)
        \item Communauté active et nombreuses librairies
        \item Courbe d'apprentissage adaptée au délai
    \end{itemize}
    \item \textbf{Options rejetées:}
    \begin{itemize}
        \item MongoDB pour données utilisateurs (manque intégrité relationnelle)
        \item Firebase (vendor lock-in, coût à l'usage)
    \end{itemize}
\end{itemize}

\textbf{Architecture 3-tiers avec flux UML :}

\begin{verbatim}
    +------------+      +-------------+      +---------------+
    |   CLIENT   | <--> |   SERVER    | <--> |   DATA        |
    |   React    |      |   Node.js   |      |   PostgreSQL  |
    |   Tier     | HTTP |   Tier      | SQL  |   Tier        |
    +------------+      +-------------+      +---------------+
    ↑                    ↑                    ↑
    |                    |                    |
    UI Events          Business Logic        Data Persistence
    State Mgmt         API Routes            Transactions
    Components         Authentication        Relations
\end{verbatim}

\textbf{Légende des flux :}
\begin{itemize}
    \item \textbf{Flux données:} Client $\rightarrow$ Server $\rightarrow$ Database $\rightarrow$ Server $\rightarrow$ Client
    \item \textbf{Flux authentification:} OAuth Bungie $\rightarrow$ Token $\rightarrow$ Session
    \item \textbf{Flux cache:} API Bungie $\rightarrow$ Redis $\rightarrow$ Client (performance)
\end{itemize}

\textbf{Schéma de données PostgreSQL (extrait) :}
\begin{lstlisting}[language=SQL]
    -- Table utilisateurs
    CREATE TABLE users (
    id SERIAL PRIMARY KEY,
    bungie_id VARCHAR(100) UNIQUE NOT NULL,
    display_name VARCHAR(50) NOT NULL,
    created_at TIMESTAMPTZ DEFAULT NOW(),
    last_login TIMESTAMPTZ,
    role user_role DEFAULT 'player'
    );
    
    -- Table escouades  
    CREATE TABLE squads (
    id SERIAL PRIMARY KEY,
    name VARCHAR(100) NOT NULL,
    leader_id INTEGER REFERENCES users(id),
    created_at TIMESTAMPTZ DEFAULT NOW()
    );
    
    -- Table sessions de raid
    CREATE TABLE raid_sessions (
    id SERIAL PRIMARY KEY,
    squad_id INTEGER REFERENCES squads(id),
    raid_name VARCHAR(100) NOT NULL,
    scheduled_at TIMESTAMPTZ NOT NULL,
    status session_status DEFAULT 'planned'
    );
\end{lstlisting}

\textbf{Stratégie de scalabilité :}
\begin{itemize}
    \item \textbf{Short-term:} Cache Redis pour API Bungie, indexation DB
    \item \textbf{Medium-term:} Read replicas PostgreSQL, CDN pour assets
    \item \textbf{Long-term:} Microservices pour fonctionnalités indépendantes
    \item \textbf{Monitoring:} Métriques performance (New Relic), alerting seuils
\end{itemize}

\textbf{Comparaison stacks techniques :}
\begin{center}
    \begin{tabular}{|p{3cm}|p{3cm}|p{3cm}|p{3cm}|}
        \hline
        \textbf{Critère} & \textbf{React/Node.js} & \textbf{Vue.js/Laravel} & \textbf{Angular/.NET} \\
        \hline
        Simplicité & \ding{72}\ding{72}\ding{72}\ding{72}\ding{72} & \ding{72}\ding{72}\ding{72}\ding{72}\ding{73}  & \ding{72}\ding{72}\ding{72}\ding{73}\ding{73}  \\
        Performance & \ding{72}\ding{72}\ding{72}\ding{72}\ding{73}  & \ding{72}\ding{72}\ding{72}\ding{72}\ding{73}  & \ding{72}\ding{72}\ding{72}\ding{72}\ding{72} \\
        Écosystème & \ding{72}\ding{72}\ding{72}\ding{72}\ding{72} & \ding{72}\ding{72}\ding{72}\ding{72}\ding{73}  & \ding{72}\ding{72}\ding{72}\ding{72}\ding{73}  \\
        Courbe apprentissage & \ding{72}\ding{72}\ding{72}\ding{72}\ding{73}  & \ding{72}\ding{72}\ding{72}\ding{73}\ding{73}  & \ding{72}\ding{72}\ding{73}\ding{73}\ding{73}  \\
        Décision & \textbf{CHOISI} & Alternative & Trop complexe \\
        \hline
    \end{tabular}
\end{center}

\section{Définition du MVP}

\textbf{Périmètre MVP - GitHub Milestones :}
\begin{itemize}
    \item \textbf{Milestone: MVP-v1.0} - Due: March 15, 2026
    \item \textbf{Épics incluses:}
    \begin{itemize}
        \item \textbf{AUTH:} OAuth Bungie, gestion sessions
        \item \textbf{GUIDES:} 3 raids détaillés (Vault, Last Wish, Deep Stone)
        \item \textbf{SQUADS:} Création, invitation, gestion basique
        \item \textbf{PLANNING:} Calendrier simple, créneaux
    \end{itemize}
    \item \textbf{Épics exclues:}
    \begin{itemize}
        \item Chat temps réel
        \item Système de badges
        \item Analytics avancées
        \item App mobile
    \end{itemize}
\end{itemize}

\textbf{Parcours utilisateurs complets MVP :}

\textbf{Parcours 1: Premier raid réussi}
\begin{enumerate}
    \item Thomas arrive sur la plateforme (landing page)
    \item Se connecte avec son compte Bungie (OAuth)
    \item Consulte le guide "Vault of Glass pour débutants"
    \item Rejoint une escouade "Bienveillante débutants"
    \item Participe à sa première session (2h)
    \item Donne son feedback sur l'expérience
\end{enumerate}

\textbf{Parcours 2: Organisation optimisée}
\begin{enumerate}
    \item Sarah se connecte (session existante)
    \item Crée une escouade "Raiders Expérimentés"
    \item Planifie une session pour samedi 20h
    \item Invite 5 coéquipiers par leurs pseudos
    \item La session apparaît dans tous les agendas
    \item Rappel automatique 1h avant
\end{enumerate}

\textbf{Plan de test de validation MVP :}
\begin{itemize}
    \item \textbf{Date:} 1-7 mai 2026
    \item \textbf{Participants:} 8 utilisateurs (3 débutants, 3 expérimentés, 2 leaders)
    \item \textbf{Scénarios testés:} 5 parcours utilisateurs critiques
    \item \textbf{Métriques:} Taux de succès, temps completion, satisfaction
    \item \textbf{Livrable:} Rapport validation avec go/no-go release
\end{itemize}

\textbf{KPI par fonctionnalité MVP :}
\begin{center}
    \begin{tabular}{|p{3cm}|p{3cm}|p{3cm}|}
        \hline
        \textbf{Fonctionnalité} & \textbf{Indicateur} & \textbf{Cible} \\
        \hline
        Authentification & Taux de succès connexion & > 95\% \\
        \hline
        Guides interactifs & Temps moyen consultation & < 25 minutes \\
        \hline
        Gestion escouades & Nombre escouades créées & 100 premier mois \\
        \hline
        Calendrier & Temps planification session & < 10 minutes \\
        \hline
    \end{tabular}
\end{center}

\section{Roadmap}

\textbf{Roadmap GitHub détaillée :}
\textbf{Milestone: POC - 5 février 2025}
\begin{itemize}
    \item Authentification Bungie fonctionnelle
    \item Mockup guides interactifs
    \item Schéma base de données validé
    \item Architecture technique finalisée
\end{itemize}

\textbf{Milestone: MVP - 15 mars 2026}
\begin{itemize}
    \item Authentification OAuth Bungie opérationnelle
    \item Guides interactifs pour 3 raids principaux
    \item Gestion basique des escouades
    \item Calendrier collaboratif simple
    \item Interface responsive web
    \item Déploiement environnement de test
\end{itemize}

\textbf{Milestone: Version 1.0 - 15 mai 2026}
\begin{itemize}
    \item \textbf{Fonctionnalités complètes:}
    \begin{itemize}
        \item Système de badges et gamification
        \item Profils joueurs détaillés avec statistiques
        \item Recherche avancée joueurs/escouades
        \item Notifications et rappels automatiques
        \item Analytics de base (tableau de bord admin)
    \end{itemize}
    \item \textbf{Performance et qualité:}
    \begin{itemize}
        \item Tests automatisés (couverture > 80\%)
        \item Temps de réponse < 2 secondes
        \item{Documentation utilisateur complète}
        \item Sécurité renforcée (audit de sécurité)
    \end{itemize}
    \item \textbf{Déploiement production:}
    \begin{itemize}
        \item Plateforme déployée en production
        \item Monitoring et alerting configurés
        \item Sauvegardes automatiques
        \item CI/CD entièrement automatisé
    \end{itemize}
    \item \textbf{Objectif utilisateurs:} 100 utilisateurs actifs
\end{itemize}

\textbf{Milestone: Version 1.1 - 15 juillet 2026}
\begin{itemize}
    \item Chat en temps réel intégré
    \item Système de recommandations d'équipements
    \item Amélioration de l'UX/UI basée sur les retours
    \item Optimisation des performances
    \item Support multilingue (anglais/français)
\end{itemize}

\textbf{Milestone: Version 1.2 - 15 septembre 2026}
\begin{itemize}
    \item Analytics avancées et rapports détaillés
    \item Intégration avec Discord webhooks
    \item Système de clans étendu
    \item Guides pour tous les raids disponibles
    \item 300 utilisateurs actifs
\end{itemize}

\textbf{Milestone: Version 2.0 - 15 janvier 2027}
\begin{itemize}
    \item Application mobile React Native
    \item API publique pour développeurs
    \item Système de streaming intégré
    \item Fonctionnalités sociales avancées
    \item 500+ utilisateurs actifs
\end{itemize}

\textbf{Indicateurs de progression :}
\begin{itemize}
    \item \textbf{Code:} Couverture tests > 80\%, dette technique < 5\%
    \item \textbf{Utilisateurs:} Croissance mensuelle > 20\%, rétention > 60\%
    \item \textbf{Performance:} Temps réponse < 2s, disponibilité > 99\%
    \item \textbf{Métier:} Réduction temps organisation < 20 minutes
    \item \textbf{Satisfaction:} Note moyenne ≥ 4.5/5 sur les guides
\end{itemize}

\section{Liens utiles}

\begin{itemize}
    \item User Stories: \url{https://www.mountaingoatsoftware.com/agile/user-stories}
    \item MoSCoW: \url{https://www.productplan.com/glossary/moscow-prioritization/}
    \item PostgreSQL Docs: \url{https://www.postgresql.org/docs/}
    \item MongoDB Modeling: \url{https://bit.ly/mongodb-modeling}
    \item Architecture 3-tier: \url{https://en.wikipedia.org/wiki/Multitier_architecture}
\end{itemize}
